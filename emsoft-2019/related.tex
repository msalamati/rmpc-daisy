% !TEX root = main.tex
\section{Related Work}

Any digital controller can be implemented using a wide range of available platforms. Large-size manufacturers use industrial digital computers called \emph{programmable logic controller (PLC)} for such implementations. A PLC is able to perform the computations required by a digital controller using floating-point arithmetic.  Low-end applications utilize microcontrollers with limited memory capacity and computational power to keep the costs of the implementation down.

A digital controller is implemented as an input-output relation or as a state-space representation.
A controller specifies a unique input-output behaviour but has uncountably many state-space representations that are mathematically equivalent.  However, the actual finite-precision implementations of these equivalent representations will result in different errors. See~\cite{ParkPSL17} for an analysis of the finite-precision error in the implementations of different state-space representations and \cite{Anta10} for a software tool that can check system stability under finite-precision implementation errors.

The problem of controller synthesis under safety requirements on the states has
been investigated mostly for Model Predictive Control (MPC)~\cite{camacho2013model} (also called receding horizon control).
Researchers have investigated designing MPC controllers for satisfying safety requirements expressed as temporal logic formulas 
\cite{FMPS18,KaramanSF08,raman2014model,WongpiromsarnTM12,pant2017smooth,kim2017dynamic}.
%,LiNSXL17, SadighK16,
\Sadegh{We can reduce the references.}
The main technique is to optimize the robust satisfaction of 
the formula~\cite{donze2010robust} (i.e., a quantitative measure of satisfaction).
These works utilize MPC an an online method that requires solving at runtime often computationally 
expensive optimization problems. In contrast, the explicit MPC used in our work performs controller synthesis at the design time.

\Sadegh{adding a few references on explicit MPC.}