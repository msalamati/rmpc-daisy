% !TEX root = main.tex
\section{Related Work}

Any digital controller can be implemented using a wide range of available
platforms. Large-size manufacturers use industrial digital computers called
\emph{programmable logic controller (PLC)} for such implementations. A PLC is
able to perform the computations required by a digital controller using
floating-point arithmetic.  Low-end applications utilize microcontrollers with
limited memory capacity and computational power to keep the costs of the
implementation down.

% most relevant work: considers both control and finite-precision
\paragraph{Verification of Finite-precision Controller Implementations}
% A digital controller is implemented as an input-output relation or as a state-space representation.
% A controller specifies a unique input-output behaviour but has uncountably many state-space representations that are mathematically equivalent.  However, the actual finite-precision implementations of these equivalent representations will result in different errors. See~\cite{ParkPSL17} for an analysis of the finite-precision error in the implementations of different state-space representations and \cite{Anta10} for a software tool that can check system stability under finite-precision implementation errors.

While control design algorithms often consider disturbance or noise, they
usually assume an ideal infinite-precision implementation of the controller.
Several works have explicitly considered the effects of finite-precision
implementations on controller robustness.

Given a precision by the user, \citet{imperialrmpc} present an algorithm which
iteratively designs a robust MPC controller. Similar to our loop, each iteration
bounds the implementation error due to fixed-point arithmetic and if it exceeds
the initial disturbance used for the design, repeats the process with an
adjusted disturbance. This approach does not optimize for the precision and
bounds implementation errors by reduction to an approximate LP problem. The
authors further note that the iterations may diverge, i.e. a controller may not
be found. Since our algorithm optimizes and adjust precision dynamically it will
eventually find an implementation.

\citet{Anta10} take an existing controller and its implementation, derive safe
bounds from the controller under which stability is guarantees and verify that
the implementation in fixed-point arithmetic satisfies this error bound.
Similarly, \citet{ParkPSL17} takes an existing floating-point controller
implementation, reconstructs the controller and verifies that roundoff errors
due to finite precision remain below a user-given error bound.

Unlike previous work which assumed a given implementation precision, the
approach presented in this paper synthesizes both the controller and the
fixed-point implementation \emph{at the same time} and thus provides
a fully automated approach.

\citet{Abate2017} design safe feedback controllers with counterexample guided
inductive synthesis (CEGIS). Safety verification needs to consider quantization
errors in the controller and in the plant model as the algorithm is based on
bounded model checking (BMC) and tracks roundoff errors with interval
arithmetic. While this algorithm generates the controller and its
implementation, due to limitations in BMC it only considers uniform fixed-point
word length in steps of 8.
\eva{I guess it also does not consider robustness? Not sure how this is related.}

\citet{memoryMPC} propose to use universal numbers (unums) instead of
traditional floating-point or fixed-point arithmetic for implementing robust
controllers, but without an error analysis. While unums can reduce memory
footprint w.r.t. a floating-point implementation, their comparison against
fixed-point arithmetic in terms of memory and performance is unclear.

% control-only
\paragraph{Controller Synthesis}
The problem of controller synthesis under safety requirements on the states has
been investigated mostly for Model Predictive Control (MPC)~\cite{camacho2013model} (also called receding horizon control).
Researchers have investigated designing MPC controllers for satisfying safety requirements expressed as temporal logic formulas 
\cite{FMPS18,KaramanSF08,raman2014model,WongpiromsarnTM12,pant2017smooth,kim2017dynamic}.
%,LiNSXL17, SadighK16,
\Sadegh{We can reduce the references.}
The main technique is to optimize the robust satisfaction of 
the formula~\cite{donze2010robust} (i.e., a quantitative measure of satisfaction).
These works utilize MPC an an online method that requires solving at runtime often computationally 
expensive optimization problems. In contrast, the explicit MPC used in our work performs controller synthesis at the design time.

\Sadegh{adding a few references on explicit MPC.}

% finite-precision only
\paragraph{Finite-Precision Optimization}
General-purpose techniques for synthesizing fixed-point implementations of 
arithmetic expressions have been developed in the space of embedded systems,
where resource efficiency is generally important.
Some work has used dynamic analyses for estimating
errors~\cite{Gaffar2004,Mallik2007}, which, however, do not provide guarantees.
Several approaches~\cite{Lee2006,Osborne2007,Kinsman2009,Pang2011} use sound
error analysis techniques, similar to the ones implemented in the tool Daisy
which we use.
