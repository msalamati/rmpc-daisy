%
% The first command in your LaTeX source must be the \documentclass command.
\documentclass[sigconf,review,anonymous]{acmart}
\usepackage{listings}
\usepackage{multirow}
\usepackage{lscape}
\usepackage{amsmath}
\usepackage{tikz}
\usetikzlibrary{shapes,arrows}
\usepackage{graphicx}

%
% defining the \BibTeX command - from Oren Patashnik's original BibTeX documentation.
\def\BibTeX{{\rm B\kern-.05em{\sc i\kern-.025em b}\kern-.08emT\kern-.1667em\lower.7ex\hbox{E}\kern-.125emX}}

\newcommand{\statevar}{x_{k}}
\newcommand{\statevarmath}{$x_{k}\,$}
\newcommand{\qstatevar}{\hat{x}_{k}}
\newcommand{\qstatevarmath}{$\hat{x}_{k}\,$}
\newcommand{\statespace}{X}
\newcommand{\regioni}[1]{$X_{{#1}}$}
\newcommand{\regionimath}[1]{X_{{#1}}}
% Rights management information. 
% This information is sent to you when you complete the rights form.
% These commands have SAMPLE values in them; it is your responsibility as an author to replace
% the commands and values with those provided to you when you complete the rights form.
%
% These commands are for a PROCEEDINGS abstract or paper.
\copyrightyear{2018}
\acmYear{2018}
\setcopyright{acmlicensed}
\acmConference[Woodstock '18]{Woodstock '18: ACM Symposium on Neural Gaze Detection}{June 03--05, 2018}{Woodstock, NY}
\acmBooktitle{Woodstock '18: ACM Symposium on Neural Gaze Detection, June 03--05, 2018, Woodstock, NY}
\acmPrice{15.00}
\acmDOI{10.1145/1122445.1122456}
\acmISBN{978-1-4503-9999-9/18/06}

%
% These commands are for a JOURNAL article.
%\setcopyright{acmcopyright}
%\acmJournal{TOG}
%\acmYear{2018}\acmVolume{37}\acmNumber{4}\acmArticle{111}\acmMonth{8}
%\acmDOI{10.1145/1122445.1122456}

%
% Submission ID. 
% Use this when submitting an article to a sponsored event. You'll receive a unique submission ID from the organizers
% of the event, and this ID should be used as the parameter to this command.
%\acmSubmissionID{123-A56-BU3}

%
% The majority of ACM publications use numbered citations and references. If you are preparing content for an event
% sponsored by ACM SIGGRAPH, you must use the "author year" style of citations and references. Uncommenting
% the next command will enable that style.
%\citestyle{acmauthoryear}

%
% end of the preamble, start of the body of the document source.
\begin{document}

%
% The "title" command has an optional parameter, allowing the author to define a "short title" to be used in page headers.
\title{Memory-efficient Mixed-Precision Implementations for Robust Explicit MPC}

%
% The "author" command and its associated commands are used to define the authors and their affiliations.
% Of note is the shared affiliation of the first two authors, and the "authornote" and "authornotemark" commands
% used to denote shared contribution to the research.

\author{Ben Trovato}
\authornote{Both authors contributed equally to this research.}
\email{trovato@corporation.com}
\orcid{1234-5678-9012}
\author{G.K.M. Tobin}
\authornotemark[1]
\email{webmaster@marysville-ohio.com}
\affiliation{%
  \institution{Institute for Clarity in Documentation}
  \streetaddress{P.O. Box 1212}
  \city{Dublin}
  \state{Ohio}
  \postcode{43017-6221}
}

\author{Lars Th{\o}rv{\"a}ld}
\affiliation{%
  \institution{The Th{\o}rv{\"a}ld Group}
  \streetaddress{1 Th{\o}rv{\"a}ld Circle}
  \city{Hekla}
  \country{Iceland}}
\email{larst@affiliation.org}

\author{Valerie B\'eranger}
\affiliation{%
  \institution{Inria Paris-Rocquencourt}
  \city{Rocquencourt}
  \country{France}
}

\author{Aparna Patel}
\affiliation{%
 \institution{Rajiv Gandhi University}
 \streetaddress{Rono-Hills}
 \city{Doimukh}
 \state{Arunachal Pradesh}
 \country{India}}


%
% By default, the full list of authors will be used in the page headers. Often, this list is too long, and will overlap
% other information printed in the page headers. This command allows the author to define a more concise list
% of authors' names for this purpose.
\renewcommand{\shortauthors}{Trovato and Tobin, et al.}

%
% The abstract is a short summary of the work to be presented in the article.
\begin{abstract}
We propose an optimization for space-efficient implementations of explicit model-predictive controllers (MPC)
for robust control of linear time invariant (LTI) systems on embedded platforms. 
We obtain an explicit-form robust model-predictive controller as a solution to a multi-parametric 
linear programming problem.
The structure of the controller is a polyhedral decomposition of the control domain,
with an affine map for each domain.
While explicit MPC is suited for embedded devices with low computational power, the memory requirements
for such controllers can be high.
We provide an optimization algorithm for a mixed-precision implementation of the controller,
where the deviation of the implemented controller from the original one is within the robustness
margin of the robust control problem.
The core of the mixed-precision optimization is an iterative static analysis that co-designs
a robust controller and a low-bitwidth approximation 
that is statically guaranteed to always be within the robustness margin to the original controller.
We have implemented our algorithm and show, on a set of Simulink benchmarks, that our optimization
can reduce space requirements by up to 20\% (and about 10\% on average), compared to a fixed precision
implementation of the original controller.


\end{abstract}

%
% The code below is generated by the tool at http://dl.acm.org/ccs.cfm.
% Please copy and paste the code instead of the example below.
%
\begin{CCSXML}
<ccs2012>
 <concept>
  <concept_id>10010520.10010553.10010562</concept_id>
  <concept_desc>Computer systems organization~Embedded systems</concept_desc>
  <concept_significance>500</concept_significance>
 </concept>
 <concept>
  <concept_id>10010520.10010575.10010755</concept_id>
  <concept_desc>Computer systems organization~Redundancy</concept_desc>
  <concept_significance>300</concept_significance>
 </concept>
 <concept>
  <concept_id>10010520.10010553.10010554</concept_id>
  <concept_desc>Computer systems organization~Robotics</concept_desc>
  <concept_significance>100</concept_significance>
 </concept>
 <concept>
  <concept_id>10003033.10003083.10003095</concept_id>
  <concept_desc>Networks~Network reliability</concept_desc>
  <concept_significance>100</concept_significance>
 </concept>
</ccs2012>
\end{CCSXML}

\ccsdesc[500]{Computer systems organization~Embedded systems}
\ccsdesc[300]{Computer systems organization~Redundancy}
\ccsdesc{Computer systems organization~Robotics}
\ccsdesc[100]{Networks~Network reliability}

%
% Keywords. The author(s) should pick words that accurately describe the work being
% presented. Separate the keywords with commas.
\keywords{datasets, neural networks, gaze detection, text tagging}
%\keywords{datasets, neural networks, gaze detection, text tagging}

%
% A "teaser" image appears between the author and affiliation information and the body 
% of the document, and typically spans the page. 
%\begin{teaserfigure}
%  \includegraphics[width=\textwidth]{sampleteaser}
%  \caption{Seattle Mariners at Spring Training, 2010.}
%  \Description{Enjoying the baseball game from the third-base seats. Ichiro Suzuki preparing to bat.}
%  \label{fig:teaser}
%\end{teaserfigure}

\maketitle

% !TEX root = main.tex
\section{Introduction}
\eva{General comment: it feels a bit long until the intro discusses the actual problem,
but it reads nicely and motivates the problem well, so it's probably fine}
\eva{General comment: should we add some more references to related work (and not only to MPC)?}
Model predictive control (MPC) is a technique to design control actions by solving finite-horizon open-loop
optimal control problems at each sampling instant.
The result of each optimization gives a sequence of optimal control actions, only the first of which is applied
to the process.
The same procedure is applied in the next time instant with a shifted time horizon and a new initial state, 
after receiving the updated values of the process state.
The optimization problem in MPC uses a dynamic model of the process, encodes all input and output (state) constraints, and 
optimizes a performance index. 
MPC has shown to be successful in a wide variety of industrial applications, due to its 
ability to systematically handle processes with many state and input variables as well as constraints on them. 

The main difference between MPC and conventional control is in the nature of the function that maps the measured outputs to control actions. 
MPC computes such a function \emph{online}, whereas a conventional controller pre-computes the function offline.
The online computations required in MPC limits its applicability to slow processes and fast computation platforms: 
the sampling time has to be large enough and the platform fast enough to allow enough time for solving the optimization problem
and obtaining the optimal action for the next time instance. 
Moreover, the optimization solver needs to be certified when using MPC in safety critical applications.

One way to tackle these problems is through \emph{explicit MPC} \cite{Bemporad:2002,Alessio2009}, which formulates the optimization
problem but computes offline a symbolic representation of the solution as a function of the state.
At run-time, the solution is evaluated on the current state as in conventional control.
For example, for linear time invariant models with linear constraints and quadratic costs, the optimization
problem for MPC can be modeled as a quadratic program, and explicit MPC techniques solve the optimization problem
using multi-parametric programming techniques.
The explicit solution is representable as a partition of the controller domain into a number of polyhedral regions,
and an affine map for each region.
The implementation of explicit MPC stores a lookup table of the affine maps and performs a number of affine computations
online to find the appropriate region for the current state and to evaluate the control.  
Explicit MPC thus expands the class of systems being controlled by MPC strategies, by taking out the need to compute fast online
or to certify a complex optimization routine.

However, there are still bottlenecks in implementing explicit MPC on resource-constrained embedded micro-controllers.
First, implementations of explicit MPCs can suffer from large memory usage, because the solution of the optimization
problem can involve many (often hundreds) of regions.
Since the memory consumption grows linearly with the number of regions, 
this can be a limited factor in using explicit MPC on resource-constrained micro-controllers.
Second, many low-end micro-controllers only support fixed-point arithmetic.
Errors in the controller implementation are inversely proportional to the number of bits used for representing each variable. 
An implementation of explicit MPC has to be \emph{robust} to 
implementation errors to be able to enforce hard constraints on the states at run time.

In this paper, we consider the problem of implementing explicit MPC on low-end microcontrollers with fixed-point arithmetic
in a \emph{memory-efficient} and \emph{robust} way.
We propose an automatic controller and mixed-precision implementation co-design technique that computes a minimal set \todo{What is meant by minimal?} of mixed
precision assignments for all variables while ensuring the resulting implementation error remains within the robustness margin.  


Our proposed method iteratively solves a robust version of the MPC which explicitly considers fixed-point implementation errors
in the model; see~\autoref{fig:overview}.
Initially, we estimate a bound $\Delta = \Delta_0$ on the implementation error and solve a
min-max quadratic program for explicit robust MPC \cite{delaPea:2005}, where the system
model has an explicit disturbance bounded by $\Delta$.
The solution is represented as a piecewise affine map.
We use an automated mixed-precision tuning tool to find an approximate implementation of the piecewise affine
map. 
We statically compute the error of the implementation, taking into account errors
in evaluating the affine maps as well as choosing the wrong affine map due to quantization.
If the error is at most $\Delta$, we are done: the implementation satisfies the robustness margin in the model.
If not, we increase the error bound $\Delta$ and run the loop again to find a new robust controller and find its best mixed-precision
implementation.

We have implemented our algorithm on top of Matlab's multi-parametric toolbox 
for robust explicit MPC \cite{Matlab-MPT?}
and the Daisy tool for multi-precision tuning and fixed-point error analysis \cite{Daisy}.
We have applied our technique on a number of standard benchmark examples.
Our implementation finds that our mixed-precision implementation can save up to 20\% memory (on average, XXX\%) in controller
implementations over a uniform-precision implementation (and a saving of XXX\% over a uniform 32-bit implementation), while
maintaining the correctness of the controller.
In absolute terms, the saving corresponds to YYY KB over a uniform precision implementation on our largest benchmark.
In each case, the analysis takes only a few minutes of computation.
We also demonstrate the scalability of our mixed-precision tuning and error bounds analysis:
we show that on explicit MPC controllers with thousands of regions, our tool finishes in a few minutes\todo{Hours?}.


\input{motiv}
% !TEX root = main.tex
\section{Background}
In this section, we provide relevant background on robust explicit MPC,
fixed-point arithmetic and error analysis.

\subsection{Robust Explicit MPC}
\def\reals{\mathbb{R}}
We consider the class of linear time-invariant (LTI) systems characterized by the difference equation
\begin{equation}
\label{eq:DSS}
x_{k+1}=Ax_k+Bu_k+Ew_k,\quad  k=0,1,2,\ldots
\end{equation}
where $x\in \reals^{n\times 1}$ is the state, $u\in \reals^{m\times 1}$ is the control input and $w\in \reals^{d\times 1}$ is the disturbance. Matrices $A\in \reals^{n\times n}$, $B\in \reals^{n\times m}$ and $E\in \reals^{n\times d}$  capture respectively the effects of current state, input and disturbance on the next state. We assume the disturbance $w$ belongs to a set $\mathcal{W}$ where $\mathcal{W}$ is a polyhedral set.
In this paper, we focus on the robust formulation of MPC which at each time step
minimizes the worst-case value of an objective function with respect to the
disturbances over the control inputs. We assume the objective function is
quadratic with respect to the states and inputs.
We assume a linear translation on the input with the form
$$u_k=\mu x_k+v_k$$
and synthesize $v_k$ instead of $u_k$. This choice reduces the 
conservativeness of the optimization and enlarges the set of feasible input
trajectories. The matrix $\mu$ is selected such that some desired property is satisfied under suitable assumptions on the system, e.g., stability if the pair $(A,B)$ is stabilizable.

%\eva{I don't understand the above sentence. Maybe split into several sentences? Where does $u_k$ come from (the sentence right now says you assume $u_k$)?}
%The modifed state evolution of the system will be $x_{k+1}=\bar Ax_k+Bv_k+Ew_k$ with $\bar A := A +BK$.
% and $v_k=\kappa(x_k)$ denotes the output of model predictive controller.  
The constrained optimization at each time step is of the form
\begin{align}
\label{eq:RMPC_prob}
J^{\ast}(x_0)=&\min_{v_0,\cdots,v_{N-1}} \max_{w_0,\cdots,w_{N-1}} \sum_{i=0}^{N-1}(x_i^TQx_i+u_i^TRu_i) + x_N^TQ_Fx_N\nonumber\\
\text{s.t.} \quad &x_{i+1}=Ax_i+Bu_i + E w_i, \quad\forall i\in\{0,1,\cdots,N-1\}\nonumber\\
\quad &u_i=\mu x_i+v_i, \quad\forall i\in\{0,1,\cdots,N-1\}\nonumber\\
&u_i\in\mathcal{U},x_i\in\mathcal{X},\quad \forall w_i\in\mathcal{W},\,\,\forall i\in\{0,1,\cdots,N\},
%&x_i\in\mathcal{X},\quad \forall w_i\in\mathcal{W},\quad\forall i\in\{0,1,\cdots,N\},
\end{align}
%\Sadegh{The role of disturbance is missing in the inequalities. We should say, we want to satisfy the inequalities for all possible values of the disturbance.}
%
where $\mathcal U$ and $\mathcal X$ are polyhedral sets denoting the feasible
sets of inputs and states. Positive definite matrices $Q\in\reals^{n\times n}$ and
$Q_F\in\reals^{n\times n}$ indicate weights on the states. $R\in\reals^{m\times m}$ is
positive semidefinite and indicates a weight on the input in the objective function. $N$ denotes the length of the prediction horizon.
%\eva{What are $Q, Q_F, R$ and where do they come from?}

\begin{theorem}[\cite{delaPea:2005}]
\label{thm:EMPC}
The optimization \eqref{eq:RMPC_prob} can be translated into a multi-parametric quadratic problem which admits a closed-form solution. Furthermore, for the case that $R>0$, the controller is $u_k = \mu x_k + \kappa(x_k)$ with $\kappa(\cdot)$ being a continuous piecewise affine (PWA) function over polyhedral regions:
\begin{equation}
\label{eq:affine_map}
\kappa(x_k)=
\begin{cases}
F_1x_k+G_1 & \text{if $x_k\in \mathcal{R}_1$}\\
F_2x_k+G_2 & \text{if $x_k\in \mathcal{R}_2$}\\
\vdots\\
F_Px_k+G_P & \text{if $x_k\in \mathcal{R}_P$}
\end{cases} 
\end{equation}
where $\mathcal{R}_i$ is a polyhedral region determined by a set of linear inequalities $\mathcal R_i = \{x\in\mathcal X\,|\,H_ix\leq K_i\}$. 
\end{theorem}
The essential idea behind this theorem is to utilize the closed form $x_i=A^ix_0+\sum_{l=0}^{i-1}A^{i-l-1}Bu_l$ and transform the optimization \eqref{eq:RMPC_prob} into the following quadratic program:
%\begin{align}
%	&\min_{U_N}U_N^TZU_N+x_0^TVU_N+x_0^TYx_0\nonumber\\
%	&s.t.\quad JU_N\leq s+Dx_0
%\end{align} 
\begin{align}
\label{eq:multi_param_prog}
J^{\ast}(x_0)=\min_{z,\gamma}& \left[x_0^TYx_0+\frac{1}{2}z^THz+\gamma\right]\\
\text{s.t.} \quad &G_mz+g_m\gamma\leq W_m+S_mx_0,\nonumber\\
&G_cz\leq W_c+S_cx_0\nonumber
\end{align}
where $z\in \reals^{mN}$ and $\gamma\in\reals$ are decision variables, and $Y$, $H$, $G_m$, $G_c$, $S_m$, $S_c$, $W_m$, $W_c$, and $g_m$ 
are matrices of proper sizes that can be easily obtained from the original optimization \eqref{eq:RMPC_prob}  (see \cite{delaPea:2005}).

%\Sadegh{Adding a remark and saying that the approach works as long as the solution of formulated optimization has a piecewise affine form?} 

The optimization \eqref{eq:multi_param_prog} can be solved using multi-parametric techniques to compute the explicit form of $\kappa(\cdot)$.
An efficient implementation of such computations is available in the multi-parametric toolbox of Matlab~\cite{matlabMPT, matlabYALMIP}.
Let us denote the set of states $x_0$ for which the optimization  \eqref{eq:multi_param_prog} is feasible by $\mathcal X_s$. Then $\mathcal X_s\subseteq \mathcal X$ and is the union of all regions $\mathcal{R}_{i}$:
\begin{equation}
\mathcal X_s = \bigcup_{i}(H_{i}\statevar\leq K_{i}).
\end{equation}


The implementation of the controller stores the matrices $F_i,G_i,H_i,K_i$ for all $i\in\{1,2,\ldots,P\}$. 
At each time step $k$, the current state $x_k$ is used to detect the polyhedral region 
such that $H_i x_k\le K_i$. Then the control action $v_k = F_i x_k + G_i$ is computed and applied to the system.
Several algorithms are proposed in the literature \cite{Mnnigmann:2011,Jones:2006} to find the right polyhedral region $i$ to which the state $x_k$ belong. 
The most straightforward technique is a linear search over all polyhedral regions (Fig.~\ref{lst:caseof}).

\begin{figure}[t]
\begin{lstlisting}[mathescape=true,basicstyle=\small\ttfamily,morekeywords={if, then, elseif, return}]
if ($H_{1}\statevar \leq K_{1}$) then $v_{k}=F_1x_k+G_1$
elseif ($H_{2}\statevar \leq K_{2}$) then $v_{k}=F_2x_k+G_2$
elseif ($H_{3}\statevar \leq K_{3}$) then $v_{k}=F_3x_k+G_3$
...
elseif ($ H_{P}\statevar\leq K_{P}$) then $v_{k}=F_Px_k+G_P$
return $v_k$
\end{lstlisting}
\caption{Structure of an explicit MPC controller}
\label{lst:caseof}
\end{figure}
%Once the polytope was determined, computing for $F_ix+G_i$ requires even less effort as it only involves summation and multiplication.  

To keep the implementation cost down, low-end microcontrollers usually have
limited memory and computational power. Therefore, the implementation of the above
explicit MPC on microcontrollers will bring some issues since all regions'
matrices ($H_i$s and $K_i$s) and activation functions ($F_i$s and $G_i$s) have
to be stored on the microcontroller. Our paper presents an
approach for minimizing the memory usage of the implementation while satisfying
the hard constraints on the states.
%\section{Finite Precision Implementation}
%\Sadegh{To be included by Rocco}

\section{Methodology}
In this work we present an innovative method for the off-line design (explicit) of robust MPC controllers for linear time invariant systems.

Consider the (discrete) state space model:
\begin{flalign}\label{eq:statespacemodel}
& x_{k+1} = A\statevar + Bu_{k}
%& y_{k} = C\statevar\nonumber
\end{flalign}

where $\statevar\in\mathbb{R}_{n}$ is the state variable at time $k$, $u_{k}$ is the scalar control input, and $A\in\mathbb{R}^{nxn}$ and $B\in\mathbb{R}^{n}$ guarantee the stability of the system at runtime.

The controller $u_{k}$ for (\ref{eq:statespacemodel}) is found by solving an off-line optimization problem. We rely on MATLAB multi-parametric toolbox to solve such optimization problem~\cite{matlabMPT, matlabYALMIP}. 
The output of the suite is a continuous piecewise affine (PWA) function, together with a partitioning of the domain of \statevarmath. Each element in the partitioning is called region.
A generic region $i$ consists in a n-polytope, uniquely identified by a set of inequalities in the form: $H_{i}\statevar<=K_{i}$. 

We call \statespace\space the union of all regions $X_{i}$ in the domain of the state variable $x_{k}$:
\begin{equation}
\statespace = \bigcup_{i}(H_{i}\statevar<=K_{i})
\end{equation}

%https://en.wikipedia.org/wiki/Polyhedron
%https://en.wikipedia.org/wiki/Polytope

Each region $i$ is associated with an activation function $u_{i,k}=F_{i}x_{k}+G_{i}$. At runtime, the value of state variable \statevarmath  is compared against the polytopes bounds: depending on the region $i$ containing \statevarmath, the corresponding activation function $u_{i}$ is computed.


A drawback in the design of explicit MPC is that both regions boundaries (H and K) and activation functions (F and G), have to be stored on the micro-controller. Usually these devices have limited memory, in the order of KBs.



This work does not rely on any specific technique to verify which region \statevarmath belongs to, so we consider a linear search over all the regions in \statespace\space, using a case-of conditional statement similar to the one in Listing \ref{lst:caseof}.

\begin{lstlisting}[escapeinside={(*}{*)},label={lst:caseof}, caption=switch for region selection]
switch((*\statevarmath*)):
case (*$ H_{1}\statevar<=K_{1}$*) then (*$u_{1}$*)
case (*$ H_{2}\statevar<=K_{2}$*) then (*$u_{2}$*)
case (*$ H_{3}\statevar<=K_{3}$*) then (*$u_{3}$*)
...
case (*$ H_{n}\statevar<=K_{n}$*) then (*$u_{n}$*)
\end{lstlisting}

At design time, the designer specifies the maximal disturbance value the controller can tolerate: we call this numerical value \texttt{delta}. 

\texttt{delta} has to be an upper bound for both errors generated by the system itself (e.g. noise from sensors) and disturbance coming from sources external to the system (e.g. friction). The controller is then designed to be robust against any kind of disturbance up to a maximal value \texttt{delta}.

Among the sources of disturbance targeting the output of the controller, our goal is to assure that the approximation error caused by the use of finite precision arithmetic is also bounded by this design property.

In particular, (i) first we want to guarantee that the finite precision implementation of the controller produces an approximation error that is at most equal to delta.

(ii) Second, we want to take advantage of this disturbance delta, and reduce the arithmetic precision used for computations and storage, while still guarantee that the approximation error is bounded by delta. Our intuition is that a greater value for delta allows for a reduced precision for computations (e.g. with respect to 32bits precision).

\subsection{Error Analysis}
Since (i) any measurement of the plant comes with some uncertainty (e.g. uncertainty from sensors) but also (ii) due to analog-digital conversion, the value of state variable \statevarmath comes with some numerical errors. The equation for \statevarmath is than defined as:
\begin{equation}
\qstatevar=\statevar + \texttt{err}
\end{equation}
where \statevarmath is the true (unknown) measure of the plant, while \qstatevarmath is the actual value.

\texttt{err} generates instability when it is time to verify which region \qstatevarmath belongs to: it can happen that \statevarmath satisfies the bounds of region $i$, but because of \texttt{err}, \qstatevarmath belongs to region $j$. This scenario becomes more intuitive when \qstatevarmath falls \texttt{close} to the border between two neighbor regions $i$ and $j$, but in general this instability depends on the magnitude of error \texttt{err}.

Assuming that all points in region $i$ might be erroneously assigned to region $j$, without any  knowledge of the actual value of error \texttt{err}, is a too wide over-approximation of the existing instability. 

Our goal is to build a feasible geometrical space around the border between two neighbor regions $i$ and $j$, where actually it is feasible that \statevarmath belongs to region $i$ but \qstatevarmath follows in region $j$ (or vice-versa).
We call this geometrical space the \texttt{tube}.

There are two main sources of error affecting the size of the \texttt{tube}: (i) the first one is caused by analog-digital conversion, happening just before the controller receives an estimation of the plant from the sensors. We call this error $\varepsilon_{A/D}$: 

\begin{equation}\nonumber
\varepsilon_{A/D}=\frac{V_{cc}}{2^{p}-1}
\end{equation}

Where $V_{cc}$ is the reference voltage of the converter (e.g. typical 5V) and $p$ represents the number of bit of the processor.

The second error affecting the size of the \texttt{tube} is caused by (ii) the quantization of region bounds in the memory of the micro-controller. We call this error $\varepsilon_{Q}$ for error quantization.

While $\varepsilon_{A/D}$ is intrinsic in the capabilities of the device, $\varepsilon_{Q}$ depends on the precision used to store the boundaries. This second error can be regulated based on a trade-off among accuracy of the storage and memory save~\cite{memoryMPC}.

The total size of the \texttt{tube} is then:
\begin{equation}\label{eq:epsilontot}
\varepsilon=\varepsilon_{A/D}+\varepsilon_{Q}
\end{equation}
\subsection{Finite Precision Implementation}
The output of the controller can be affected by two main errors: (i) the controller chooses the wrong activation function because of $\varepsilon$ in (\ref{eq:epsilontot}), and (ii) the approximation error deriving from finite precision arithmetic used to compute the activation function itself~\cite{imperialrmpc}.

The effect of picking the wrong activation function are similar to instability in the switch reported in Listing \ref{lst:caseof}. 

In an hypothetic scenario where all the measurements were done in infinite precision arithmetic and without any uncertainty, we assume the controller $i$ would be activated. Instead, because we cannot rely on infinite precision, controller $j$ is selected. 

The error committed is the difference between the output of the two branches $i$ and $j$.

The system has to be robust against a mistake in choosing controller $i$ instead of $j$ (or vice-versa), only when \qstatevarmath falls into the \texttt{tube} between $i$ and $j$, and not for all the points in the two regions. Moveover, we are interested in the worst case scenario where is maximized the difference between the activation functions of any two neighbor regions.

In the following, we assume $i$ and $j$ are the index of two generic regions in \statespace:
\begin{flalign}
\label{eq:maximization}
&\max_{\forall i,j\;|\;neighbour(i,j)}|u_{i}-u_{j}| = \\
&\max_{\forall i,j\;|\;neighbour(i,j)}|F_{i}\statevar+G_{i} - (F_{j}\statevar+G_{j})|\nonumber
\end{flalign}
where \statevarmath belongs to the \texttt{tube} between region $i$ and $j$.
Because of the linearity of the function $u_{i}-u_{j}$, and because of the convexity of the regions $i$ and $j$, it is enough to evaluate function (\ref{eq:maximization}) at the corner points of the \texttt{tube}, instead of solving a maximization problem.

%The geometrical space where $\hat{X}$ activates $U_{i}$, while X would activate $U_{j}$ can be bounded thanks to a preliminary knowledge of the error $\varepsilon$: the distance between region $i$ and $j$ such that:
%\begin{equation}
%\hat{X}-X <= \varepsilon
%\end{equation}
%and $\hat{X}$ belongs to region $i$ but $X$ to $j$ (or vice-versa).

We compute (\ref{eq:maximization}) for all pairs of neighbor regions $i$ and $j$. Two regions are neighbors when they share at least a border.

In formula $\exists\; m,n \;$such that:
\begin{equation}
(H_{i}\statevar-K_{j})_{m} = (H_{j}\statevar-K_{j})_{n}
\end{equation}
where $m$ and $n$ are the index of the two matching borders. We label with $border_{i,j}$ the matching border shared among the two regions.

Starting from the equation of $border_{i,j}$, we delimit the geometrical space where it makes sense to compute (\ref{eq:maximization}) with: 
\begin{equation}
\begin{aligned}
(border_{i,j} >= -\varepsilon) \land
(border_{i,j} <= \varepsilon)
\end{aligned} 
\end{equation}
where $\varepsilon$ is defined in (\ref{eq:epsilontot}). We call such geometrical space the \texttt{tube} between $i$ and $j$.

When \qstatevarmath belongs to the \texttt{tube}, it might be that $u_{i}$ is activated instead of $u_{j}$ (or vice-versa). Otherwise, when \qstatevarmath does not belong to the tube, no matter the error $\varepsilon$, the right activation function is activated, and we consider only the error deriving from the computation itself.

Since computing $u_{i}$ introduces approximation error because of finite precision arithmetic, the system has to be robust against an error that is:

\begin{equation}\label{eq:fperror}
err(u_{i})_{p}=|(\hat{F}_{p}-F)\qstatevar+(\hat{G}_{p}-G)|
\end{equation}

where $\hat{F}$ and $\hat{G}$ represent the rounded values (in p bits) for the infinite precision values $F$ and $G$.

In (7) the domain of \qstatevarmath are all the points in region $i$, but also all the values in the \texttt{tube} between region $i$ and any of the neighbors of $i$. Even if some points in the tube do not belong to region $i$, it might that $u_{i}$ is (erroneously) activated.

We compute (\ref{eq:fperror}) for all regions in \statespace\space and we take the maximum value of the error (worst case):

\begin{equation}\label{eq:maxfperror}
\max_{\forall \regionimath{i}\;in\;\statespace} err(u_{i})_{p}
\end{equation}


The disturbance \texttt{delta}  has to be an upper bound to the summation of (\ref{eq:maximization}) and (\ref{eq:maxfperror}).

In formula:
\begin{flalign}
\label{eq:delta}
&delta >= \\
&\max_{\forall i,j\;|\;neighbour(i,j)}|u_{i}-u_{j}| + \max_{\forall\;\regionimath{i}\;in\;\statespace} err(u_{i})_{p}\nonumber
\end{flalign}

\subsection{Precision Tuning}
In formula (\ref{eq:epsilontot}), the equation for $\varepsilon$ contains: $\varepsilon_{A/D}$ that is the error introduced by the analog-digital conversion, and $\varepsilon_{Q}$ that is generated by the quantization of hyperplanes $H\statevar<=K$ in a finite number of bits.
This $\varepsilon_{Q}$ can be tuned based on memory availability in the micro-controller.
The rule for $\varepsilon_{Q}$ is the following:
\begin{equation}\label{eq:quantizationlines}
\varepsilon_{Q} > (H-\hat{H}_{p})\qstatevar+(K-\hat{K}_{p})
\end{equation}
This inequality assures that the distance between any hyperplane represented in infinite precision (H and K), and its counter-part quantized in p bits ($(\hat{H})_{p}$ and $(\hat{K})_{p}$), is bounded by $\varepsilon_{Q}$. 

Our goal is to solve (\ref{eq:quantizationlines}) with respect to p: assign an arbitrary value to $\varepsilon_{Q}$ and find the minimal-precision p such that the inequality holds.

The main advantage in this approach (compared to fixing the precision p a priori), is that the value of p can be tuned based on the memory availability in the micro-controller.

To solve (\ref{eq:quantizationlines}) we used Daisy: a static analyzer for finite precision expressions, ables to provide a sound upper-bound to the maximal approximation error of a formula with respect to its real (infinite precision) counterpart. Since the magnitude of the round-off error depends on the range of input variables, any variable encoded in Daisy has to be bounded. 

Since the state variable \statevarmath is bounded by \statespace, and F,G,H, and K are vectors of constants, we can rely on Daisy for this verification step.

In a similar way, we use Daisy to solve the inequality in (\ref{eq:delta}) with respect to p:
\begin{flalign}
\label{eq:deltaminusmax}
&delta - \Big(\max_{\forall i,j\;|\;neighbour(i,j)}|u_{i}-u_{j}|\Big)>=\\
& \max_{\forall\;\regionimath{i}\;in\;\statespace} err(u_{i})_{p}\nonumber
\end{flalign}

Be aware that the left side of the inequality is not parametrized in p. Only the right size of (\ref{eq:deltaminusmax}) actually depends on the precision p. After we come up with a numerical value for the left side of the inequality, we encode (\ref{eq:deltaminusmax}) in Daisy in the same way done for (\ref{eq:quantizationlines}).

\subsection{Algorithm}

\begin{lstlisting}[language=Python,numbers=left,numbersep=3pt,frame=lines,keepspaces=true,escapeinside={(*}{*)},caption=design of robust MPC with verification and precision tuning,label={lst:alg}]
delta=input()
(*$\varepsilon_{Q}$*)=input()
assert (delta>=0 and (*$\varepsilon_{Q}$*)>0)
(*$\varepsilon$*)=(*$\varepsilon_{A/D}$*)+(*$\varepsilon_{Q}$*)

while True:
design_robust_MPC(delta)
maxUij = compute (*(\ref{eq:maximization})*) with size(tube)=(*$\varepsilon$*)
if delta > maxUij:
max_err=delta-maxUij
UNI_MIX_precision(F,G,max_err)
UNI_MIX_precision(H,K,(*$\varepsilon_{Q}$*))
break
else:
delta=maxUij+(*$\varepsilon_{SAFE}$*)
\end{lstlisting}
In Listing \ref{lst:alg} we describe the procedure used to design a robust MPC controller such that (\ref{eq:delta}) is respected.

First, the designer fixes the initial values for delta and $\varepsilon_{Q}$.

Even if it possible to assign value zero to delta~\cite{imperialrmpc}, we do not encourage such initialization value: it is going to fail the analysis (at least) for the first iteration because the error in (\ref{eq:maxfperror}) is going to be greater than zero (in the remote scenario where  (\ref{eq:maximization}) is equal to zero). A better initialization would be to set delta to an arbitrary small value, slightly greater than zero.

On the other hand, the value for $\varepsilon_{Q}$ has to be strictly greater than zero, otherwise we could rely on infinite precision for p in (\ref{eq:quantizationlines}).

Once input parameters are verified, we use MATLAB toolbox to design a controller with robustness value equal to delta. 

Then, we compute (\ref{eq:maximization}) and we compare the result with delta: we are aware that the computation of (\ref{eq:maximization}) is not exact (even if it is done in 64bits precision), but usually the computation of (\ref{eq:maximization}) results in a value that is several orders of magnitude greater than the error of the computation itself. For this reason we consider this approximation error negligible from the point of view of our analysis. 

Again, in the conditional inside the loop, delta has to be strictly greater than \texttt{maxUij} otherwise we do not have space for computing the precision for (\ref{eq:deltaminusmax}).

In case the conditional statement is verified,
the precision tuning phase starts.
For the precision tuning of activation functions, we allow an error that is bounded by \texttt{max\_err}. This is exactly what is described in (\ref{eq:deltaminusmax}).

On the other hand, the quantization of polytopes borders (the region bounds in \statespace) can produce an error that is at most $\varepsilon_{Q}$, in this way we satisfy 
(\ref{eq:quantizationlines}).

In case the conditional fails and \texttt{maxUij} is greater than (or equal to) delta, the controller needs to be re-designed with a robustness value that is at least equal to the current value of \texttt{maxUij}. The same consideration done for the initialization of input parameters holds also here: the constant $\varepsilon_{SAFE}$ is used to relax the value for delta, to a value slightly greater than \texttt{maxUij}. In this way, we aim to give some space to the precision tuning phase in the next iteration of the loop. Otherwise, in case $\varepsilon_{SAFE}$ is equal to zero, the next iteration of the precision tuning phase is going to require unnecessarily wide precision value for p, usually greater than 32 bits. 
We remark that the point of the analysis is to find a low precision configuration for bounds and activation functions: in case the analysis outputs a precision greater than 32bits, we fail in our goal.
Then, with a minimal alteration to the upper bound of delta, we sensibly reduce the precision needed for $F$ and $G$, with respect to the standard 32 bits precision.



%!TEX root = main.tex
\section{Experimental Results}\label{sec:experiments}
\begin{table*}[p]
  \centering
  \caption{Inverted Pendulum and Aircraft.
  \textmd{ 
  Memory requirements in number of bits for storing F, G and H, K for uniform 32
  bit precision (Uni32), uniform custom precision (Uni, word length chosen in
  parentheses) and mixed precision (Mix), for different values of $\Delta$ and
  $\varepsilon_Q$.
  The error due to choosing a wrong region is denoted as `max' and the finite-precision
  error bound in evaluating the control actions as $err(u_{i})$.
  \%32vsU is the percentage of memory saved using Uni compared to the baseline Uni32,
  and \%UvsM is the percentage of memory saved using Mix compared to Uni.
  % delta is the disturbance used to design the controller, and $\varepsilon$ the
  % size of the safe space between two generic regions (\texttt{tube}). We fix
  % $\varepsilon$ and try different delta (in red) and vice-versa. The maximal
  % error due to a wrong activation function is \maxUij and the error bound for
  % finite precision activation functions $err(u_{i})$. F, G and H, K represent
  % the memory requirements for activation functions, and hyperplanes. Uni32 is
  % the total number of bits for the controller using 32 bits precision. Uni is
  % the uniform precision found by our analysis, together with the format. Mix is
  % the total number of bits for the controller in mixed-precision. \%32vsU is the
  % difference between the baseline Uni32 and Uni (in percentage), then \%UvsM is
  % the difference between Uni and Mix.
  }}
  \label{tab:ipd}
  \renewcommand{\arraystretch}{1.2}
  \setlength{\tabcolsep}{0.7em} % for the horizontal padding
  \begin{tabular}{l|rr|rrrrr|rrrrr}
    \toprule
     & & & \multicolumn{5}{c}{F and G} & \multicolumn{5}{|c}{H and K} \\
    %\cline{2-13}
    %\cline{5-6}
    %\cline{9-12}
    %\cline{13-16}
    
    \multirow{14}{*}{\rotatebox{90}{pendulum}} &  
    \multicolumn{1}{c}{$\Delta$}&
    \multicolumn{1}{c}{$\varepsilon_Q$} &
    %\multicolumn{1}{c}{max} &
    %\multicolumn{1}{c}{$err(u_{i})$} &
    \multicolumn{1}{|c}{Uni32}&
    \multicolumn{1}{c}{Uni}&
    \multicolumn{1}{c}{Mix}&
    \multicolumn{1}{c}{\%32vU}&
    \multicolumn{1}{c}{\%UvM}&
    \multicolumn{1}{|c}{Uni32}&
    \multicolumn{1}{c}{Uni}&
    \multicolumn{1}{c}{Mix}&
    \multicolumn{1}{c}{\%32vU}&
    \multicolumn{1}{c}{\%UvM} \\
    
    \midrule

    & \textbf{0.30} & 0.001   & 2688 & 924 (p=11) & 759 & 65.6\% & 17.9\% & 11136 & 5568 (p=16) & 4991 & 50.0\% & 10.4\% \\
    
    & \textbf{0.20} & 0.001   & 2688 & 924 (p=11) & 806 & 65.6\% & 12.8\% & 11136 & 5568 (p=16)& 4992 & 50.0\% & 10.3\% \\
    
    & \textbf{0.10} & 0.001   & 2688 & 1008 (p=12) & 908 & 62.5\% & 9.9\% & 11136 & 5568 (p=16)& 4992 & 50.0\% & 10.3\% \\
    
    & \textbf{0.08} & 0.001   & 2688 & 1092 (p=13) & 946 & 59.4\% & 13.4\% & 11136 & 5568 (p=16) & 4992 & 50.0\% & 10.3\% \\
    
    & \textbf{0.05} & 0.001   & 2688 & 1176 (p=14) & 1030 & 56.3\% & 12.4\% & 11136 & 5568 (p=16) & 4992 & 50.0\% & 10.3\% \\ 
    
    % Empy Line
    & & & & & & & & & & & &\\
    % Empy Line
    
    & 0.1 & \textbf{0.0006}  & 2688 & 1008 (p=12) & 891 & 62.5\% & 11.6\% & 11136 & 5916 (p=17) & 5261 & 46.9\% & 11.1\% \\
    
    & 0.1 & \textbf{0.0008}  & 2688 & 1008 (p=12) & 901 & 62.5\% & 10.6\% & 11136 & 5568 (p=16) & 5135 & 50.0\% & 7.8\% \\
    
    & 0.1 & \textbf{0.0010}  & 2688 & 1008 (p=12) & 908 & 62.5\% & 9.9\% & 11136 & 5568 (p=16) & 4992 & 50.0\% & 10.3\% \\
    
    & 0.1 & \textbf{0.0030}  & 2688 &  1092 (p=13) & 993 & 59.4\% & 9.1\% & 11136 & 4872 (p=14) & 4462 & 56.3\% & 8.4\% \\
    
    & 0.1 & \textbf{0.0050}  & 2688 & 1680 (p=20) & 1527 & 37.5\% & 9.1\% & 11136 & 4872 (p=14) & 4204 & 56.3\% & 13.7\% \\
    
    \midrule

    \multirow{12}{*}{\rotatebox{90}{aircraft}}
    & \textbf{0.30} & 0.001  & 9984 & 6864 (p=22) & 6210 & 31.3\% & 9.5\% & 79872 & 64896 (p=26) & 53059 & 18.8\% & 18.2\% \\
    
    & \textbf{0.20} & 0.001  & 10368 & 7452 (p=23) & 6725 & 28.1\% & 9.8\% & 82944 & 67392 (p=26) & 55098 & 18.8\% & 18.2\% \\
    
    & \textbf{0.10} & 0.001  & 10368 & 7776 (p=24) & 7134 & 25.0\% & 8.3\% & 82944 & 67392 (p=26) & 55098 & 18.8\% & 18.2\% \\
    
    & \textbf{0.08} & 0.001  & 10368 & 7776 (p=24) & 7275 & 25.0\% & 6.4\% & 82944 & 67392 (p=26) & 55098 & 18.8\% & 18.2\% \\
    
    & \textbf{0.05} & 0.001  & 10368 & 8424 (p=26) & 7840 & 18.8\% & 6.9\% & 82944 & 67392 (p=26) & 55098 & 18.8\% & 18.2\% \\
    
    % Empy Line
    & & & & & & & & & & & &\\
    % Empy Line
    
    & 0.1 & \textbf{0.0006}  & 10368 & 7776 (p=24) & 7047 & 25.0\% & 9.4\% & 82944 & 69984 (p=27) & 55705 & 15.6\% & 20.4\% \\
    
    & 0.1 & \textbf{0.0008}  & 10368 & 7776 (p=24) & 7125 & 25.0\% & 8.4\% & 82944 & 69984 (p=27) & 57859 & 15.6\% & 17.3\% \\
    
    & 0.1 & \textbf{0.0010}  & 10368 & 7776 (p=24) & 7134 & 25.0\% & 8.3\% & 82944 & 67392 (p=26) & 55098 & 18.8\% & 18.2\% \\
    
    & 0.112* & \textbf{0.0030}  & 10368 &  9720 (p=30) & 9051 & 6.3\% & 6.9\% & 82944 & 64800 (p=25) & 52754 & 21.9\% & 18.6\% \\
    
    & 0.185* &\textbf{0.0050}  & 10368 & 9720 (p=30) & 9051 & 6.3\% & 6.9\% & 82944 & 62208 (p=24) & 47877 & 25.0\%& 23.0\% \\
    \bottomrule
  \end{tabular}
\end{table*}
\begin{table*}
  \centering
  \caption{Double Integrator.\textmd{ $N$ is the prediction horizon in RMPC, time gives the execution time in minutes, Regs is the number of regions of the controller with Hyps hyperplanes. Uni32 is the total number of bits when all operations are in 32 bits, Uni the minimal uniform precision required, Mix is mixed-precision, \%32vU and \%UvM give the improvements of uniform and mixed precisions.}}
  \label{tab:di}
  \renewcommand{\arraystretch}{.2}
  \setlength{\tabcolsep}{-0em} % for the horizontal padding
  \begin{tabular}{lccc|lcccl|lcccl}
    \toprule
    \multicolumn{4}{c}{} & \multicolumn{5}{|c|}{$F$ and $G$} & \multicolumn{5}{c}{$H$ and $K$} \\
    %\hline
    %\cline{5-6}
    %\cline{9-12}
    %\cline{13-16}
    \multicolumn{1}{c}{$N$\:}&
    \multicolumn{1}{c}{time\:}&
    %\multicolumn{1}{c}{Eps} &
    \multicolumn{1}{c}{Regs} &
    \multicolumn{1}{c}{Hyps} &
    %\multicolumn{1}{c}{\maxUij} &
    %\multicolumn{1}{c}{Err-Bound} &
    \multicolumn{1}{|c}{Uni32}&
    \multicolumn{1}{c}{Uni}&
    \multicolumn{1}{c}{Mix}&
    \multicolumn{1}{c}{\%32vU\;}&
    \multicolumn{1}{c}{\%UvM}&
    \multicolumn{1}{|c}{Uni32}&
    \multicolumn{1}{c}{Uni}&
    \multicolumn{1}{c}{Mix}&
    \multicolumn{1}{c}{\%32vU\;}&
    \multicolumn{1}{c}{\%UvM} \\
    \midrule
    2 & 2 & 9 & 72 & 1728 & 810 (p=15) & 628 & 53.1\% & 22.5\% & 13824 & 7776 (p=18) & 7280 & 43.8\%& 6.4\% \\
    5 & 9 & 53 & 424 & 10176 & 5088 (p=16) & 3623 & 50.0\% & 28.8\% & 81408 & 45792 (p=18) & 42656 & 43.8\% & 6.8\% \\
    8 & 23 & 143 & 1144 & 27456 & 13728 (p=16) & 9864 & 50.0\%  & 28.1\% & 219648 & 123552 (p=18) & 114948 & 43.8\% & 7.0\% \\
    11 & 47 & 277 & 2216 & 53184 & 26592 (p=16) & 18980 & 50.0\% & 28.6\% & 425472 & 239328 (p=18) & 222616 & 43.8\% & 7.0\% \\
    
    14 & 73 & 431& 3446& 82752& 41376 (p=16)& 28685& 50.0\% & 30.7\% & 661632& 372168 (p=18)& 346020& 43.8\% &7.0\% \\
    
    17 & 106 & 621 & 4968 & 119232 & 59616 (p=16) & 40503 & 50.0\% & 32.1\% & 953856& 536544 (p=18)& 498668& 43.8\% & 7.1\% \\
    
    21 & 150 & 928 & 7432 & 178368 & 89184 (p=16) & 59409 & 50.0\% & 33.4\% & 1426944 & 802656 (p=18) & 745936 & 43.8\% & 7.1\%  \\
    
    25 & 223 & 1299 & 10392 & 249408 & 124704 (p=16) & 81889 & 50.0\%& 34.3\%& 1995264 & 1122336 (p=18)& 1043456 & 43.8\%& 7.0\% \\
    
    40 & 314 & 1829& 14632\;& 351168\:& 175584 (p=16)& 113979& 50.0\%& 35.1\%& 2809344\;& 1580256 (p=18)& 1469834& 43.8\%& 7.0\% \\
    
    %25 & 445 &1299 & 23382 & 249408 & 109116 (p=14) & 80524 & 56.3\% & 26.2\% & 2992896 & 1589976 (p=17) & 1355675 & 46.9\% & 14.7\% \\
    \bottomrule
  \end{tabular}
\end{table*}

We evaluate a prototype implementation of the algorithm in~\autoref{lst:alg}
on three examples. For the first two, we apply the complete pipeline (design and memory optimization)
which returns an end-to-end robust controller. With the third example, we evaluate
the scalability of our approach when the number of regions and hyperplanes are in
the order of tens of thousands.

% This section reports experiments with the prototype described
% in~\autoref{lst:alg} applied to three different benchmarks: in the former two,
% we evaluate the complete pipeline (design and memory optimization) to produce in
% output a complete controller. In the last experiment, we want to show the
% scalability of our approach when the number of regions and hyperplanes are in
% the order of tens of thousands.

% For the design of controllers, we rely on MATLAB MPC toolbox. The outputs of the
% suite are vectors F, G and H, K for activation functions and hyperplanes
% equations. We encode both of them in Daisy for finite-precision optimization.
The design of end-to-end robust controllers has been performed on a laptop with Intel i7-6700HQ CPU at
2.60GHz, with 16GB of RAM. The evaluation of the last benchmark runs on a
cluster with 48 Intel Xeon v2 @ 3.00GHz cores with 1TB of RAM, of which our analysis
only used 15GB.
%Note that the memory consumption of our analysis never exceed 15GB of memory.

\subsection{End-to-End Robust Controller}
%\eva{Use the same symbol for Delta and Eps as used in the technical section.}

We evaluate our complete pipeline on two benchmarks. 
The first one is the inverted pendulum problem depicted in Section \ref{sec:example}, where we set
$m=0.344$ kg, $b=0.48$ N s/m, $L=1.703$ m and $T_s=0.1$ s. The gain $K$ is
selected such the $\bar A$ has poles at $-0.1$ and $-0.5$. Moreover, we select
$N=2$, $R=1$ and $Q=Q_F=100\mathbf{\it I}$, where $\mathbf{ \it I}$ denotes the
identity matrix of proper size. 

Our second benchmark is a well-known 4D example for aircraft
controller design~\cite{Kapasouris:1998}.
The control inputs for the aircraft 4-D model are the elevator and flaperon angles,
and the attack and pitch angles are the output states that need to be regulated.
The open-loop system is unstable as it has a pole with positive real part. Both
control inputs are constrained between $\pm25$ degrees. The outputs are
only constrained during the first prediction horizon. You also specify scale
factors for outputs. Using the gain $K$, the poles of $\bar A$ are placed at
$-5$, $-3$, $-1$ and $-2$. The robust MPC problem is solved for $N=2$,
$R=\mathbf{\it I}$ and $Q=Q_F=5\mathbf{\it I}$. Note that for both of the
examples, matrices $R$, $Q$ and $Q_F$ are selected such that convergence to the
origin is given more weight compared to the control effort as long as
constraints are satisfied.


%the feasible domain $X$ is bounded in the range $X_{0} \in
%(-3.23, 3.23)$ and $X_{1}\in (-1.46, 1.46)$. MATLAB designs a controller with 14
%regions and 58 hyperplanes for each combination of delta and $\varepsilon$.
%% delta spans the interval from 0.05 to 0.3, while $\varepsilon$ from 0.0006 to 0.0050.  % it's in the table
%For the aircraft problem, the domain of X is bounded in the range $X_{0} \in
%(-10000, 10000)$, $X_{1}\in (-734, 734)$, $X_{2}\in (-10000, 10000)$ and
%$X_{3}\in (-790, 790)$. MATLAB returns a controller with 27 regions and 217
%hyperplanes, except when delta=0.30 and $\varepsilon$=0.001 where the controller
%has 26 regions and 208 hyperplanes. In general, when the disturbance value
%increases, the state domain X shrinks to be robust against a greater
%disturbance, or the number of regions reduces.

We do not compare against the existing technique of~\citet{imperialrmpc} which aims to reduce memory
usage in explicit MPC control, as it requires the user to provide a (uniform)
fixed-precision up front. Instead, we let Daisy find the minimum uniform
precision needed. Additionally, we compare against a uniform 32-bit precision
baseline, which in the absence of special user insight would be a reasonable safe choice.

\autoref{tab:ipd} shows the total number of bits required to implement each
controller using different precision options: uniform 32 bit precision
(`Uni32'), minimal uniform precision (`Uni', chosen precision in parentheses)
and mixed precision (`Mix'). We split the memory requirement into the bits
required for storing F and G and H and K. We show the results for each benchmark
for ten different combinations of $\Delta$ and $\varepsilon_Q$, varying one while
keeping the other fixed.
For the pendulum, the execution time of the whole analysis is 10 minutes no matter the
initial values for $\Delta$ and $\varepsilon_Q$, while for the aircraft it is 36
minutes. 


%\rocco{maybe table for execution times}
% In \autoref{tab:ipd} we report the evaluation of our approach on both the
% inverted pendulum problem described in the motivation section, and a well-known
% 4D example for aircraft controller design. (Describe aircraft). For each
% benchmark, we report the results for ten different combinations of delta and
% $\varepsilon$. We divided them in two halves: in the  first, we fix the value
% for delta and we try different $\varepsilon$ combinations, while in the second
% halve we do the opposite.

% \eva{In the following, I'd discuss the two examples together and structure the 
% discussion as follows:
% \begin{itemize}
% 	\item give the constant info (number of regions etc.) for both benchmarks
% 	\item explain the variation of delta and eps (ideally with a motivation for why this is interesting)
% 	\item discuss the improvements we get (i.e. first discuss the metric we care about most)
% 	\item discuss other interesting details
% \end{itemize}
% Discussing the two examples together let's us compare their results easier, i.e.
% point out differences or commonalities.}

%\eva{Fix Table 1 caption. It's not the difference, but savings. Difference is absolute.}
%\eva{Compute the actual averages for all F, G, H, K and not only approximations}
For the inverted pendulum, minimal uniform precision saves on average $57.5\%$ of memory
compared to a uniform 32 bit baseline overall, i.e. for F, G, H and K together. 
Mixed-precision further reduces the memory
requirement by $10.4\%$ on average with respect to the minimal uniform baseline. 
\autoref{tab:ipd} shows a more detailed breakdown of the memory requirements and savings
between F and G and H and K.
The memory requirements for the storage of hyperplanes H and K depends only on
the size of the tubes $\varepsilon_Q$ so that memory requirements remain constant
for fixed $\varepsilon_Q$.

%\eva{Compute actual and overall averages here too}
For the aircraft example, minimal uniform precision saves on average $33.5\%$
overall w.r.t. a uniform 32 bit baseline, and mixed-precision saves an
additional $17.6\%$ w.r.t. minimal uniform precision. We observe higher relative
memory savings by mixed-precision for storing H and K than for the inverted pendulum example.

% we save 20\% (in average) of memory when activation functions
% are in minimal uniform precision, and an additional 7\% (in average) when we use
% mixed precision. For the storage of hyperplanes equations, we save more than
% 19\% of memory (in average) from uniform precision in 32bits to minimal uniform
% precision, and an additional 19\% (in average) from uniform to mixed precision.

% We use Matlab's MPT toolbox~\cite{matlabMPT} to compute the
% explicit MPC, while underlying computations rely on YALMIP(\cite{matlabYALMIP}).
%We evaluate our algorithm with different choices of $\varepsilon_Q$ and $\Delta$.
For the inverted pendulum Matlab's MPT toolbox~\cite{matlabMPT} computes a controller
consisting of $14$ regions with $58$ 2D hyperplanes in total for all choices of $\varepsilon_Q$ and $\Delta$. 
For the aircraft model, Matlab computes a robust EMPC with
$27$ regions and $217$ 4D hyperplanes for most values of $\varepsilon_Q$ and $\Delta$. 
The only exception is when $\Delta=0.30$
and $\varepsilon_Q=0.001$ for which we get $26$ regions having $208$ hyperplanes.
In general, we expect that increasing $\Delta$ results in shrinking of feasible
set size.

%We noticed the following common behavior in the two benchmarks: 
As expected, when $\Delta$ decreases, the control actions (F, G) need to be implemented
more precisely and require more memory, because the space for approximation error
is reduced. Similarly, when the value of $\varepsilon_Q$ increases, the memory
requirements for H, K can be relaxed.

%We note two differences among these two benchmarks.
We note that for the aircraft example,
the precision for F and G is almost double with respect to the pendulum. This is
because the magnitude of F and G is on the order of $10^{3}$ while for the
pendulum it is on the order of several units. Note, however, that for F and G the memory gain
from minimal uniform to mixed precision (\%UvM) is only slightly less than the one for the pendulum.


%The second difference consists in the last two lines in Table\ref{tab:ipd}.
For the aircraft example, when $\varepsilon_Q=0.0030$, the error due to selecting the wrong region (\maxUij) 
exceeds the given value of $\Delta=0.1$ and our algorithm
needs to design a new controller (corresponding to the \texttt{No} branch in~\autoref{fig:overview}). 
The loop converges after 5 iterations (\texttt{Yes}
branch of the diagram) with $\Delta=0.112$. In each iteration, the value
for $\Delta$ is increased by $\varepsilon_{SAFE} = 10^{-3}$.
Thus, the algorithm reduces memory demand
at the expense of slightly more disturbance for the controller. When
$\varepsilon_Q=0.0050$ the analysis converges after 4 iterations.
\eva{Does this mean that for the other settings, the loop in~\autoref{fig:overview}
is executed only once?}


% \eva{It is still not clear to me whether we need this discussion or the columns `max' and `$err(u_i)$'
% in the table. These are internal computations and it seems that they do not really show anything
% surprising/different than the number of bits show?}
% The first halves of both pendulum and aircraft sections in \autoref{tab:ipd},
% show how $\Delta$ does not affect the maximal error among regions \maxUij. We
% verify that a variation for $\Delta$ (typically in the order of $\pm$ 0.1)
% results in a minimal alteration to the domain of regions X (in the order of $\pm
% 10^{-10}$), not enough to produce a sensitive variation to \maxUij. Then, when
% \maxUij\space is constant the error bound for computations $err(u_{i})$
% decreases together with $\Delta$. The memory required for the storage of F and G
% depends on $err(u_{i})$: the smaller the value for $err(u_{i})$, the more
% precise has to be the arithmetic for activation functions.

% In the second halves in \autoref{tab:ipd}, we fix the value for $\Delta$ and we
% try different $\varepsilon_Q$. When we increase the safe space between two regions
% by $\varepsilon_Q$, the value for \maxUij is computed for new corner points. It
% happens because of the continuity of PWA activation functions: when
% \statevarmath is on the border between two generic regions $i$, $j$ the value
% for \maxUij is close to zero. The more $\varepsilon_Q$ moves \statevarmath far
% from the border, the more \maxUij increases. When \maxUij is almost equal to
% $\Delta$, the space for $err(u_{i})$ is minimized, and the controller requires
% high precision for computations (see last lines for pendulum and aircraft in
% Table\ref{tab:ipd}). On the other hand, when $\varepsilon_Q$ increases, the
% analysis can reduce the memory requirements for the storage of borders (see
% columns Uni and Mix in the second halves of both benchmarks).




\subsection{Scalability}
%\Mahmoud{
The goal of the experiment in this section is to show that our algorithm works
well even for the case that the controller consists of thousands of regions.
%Using the algorithm described in Section \ref{???}, the minimum number of bits for storing the controller is computed such that $\delta$ and $\varepsilon$ are respected. 
This experiment is different from the end-to-end case in the sense that
designing robust EMPC for the error bound $\Delta$ is replaced with designing
EMPC that might not account for $\Delta$. Given an EMPC, one can perform
robustness analysis to come up with an input error bound $\Delta$ under which
the performance specifications are satisfied. From the implementation point of
view, this helps us to generate controllers with thousands of regions,
skiping the restrictions for designing robust EMPC with longer time horizons. 
% At this point, our algorithm takes $\Delta$ (along with a properly selected
% $\varepsilon_Q$) and produces the optimum mixed precision that results in
% significant memory save.

The benchmark in this experiment is the double integrator, a canonical example
of a second order control system \todo{cite something?}. The state space description
for the discrete time version of the double integrator is given by:
\begin{equation}
x_{k+1}=
\begin{bmatrix}
1 & T_s\\
0& 1	
\end{bmatrix}
x_k+
\begin{bmatrix}
0\\
T_s
\end{bmatrix}u_k
\label{eq:int2_ss}
\end{equation}
where, $T_s=0.1$ is the sample time. The state and input constraints are
\begin{align*}
	\begin{bmatrix}
		-5\\-5
	\end{bmatrix}\leq&
	x_k\leq
	\begin{bmatrix}
	5\\5
	\end{bmatrix}, \quad
	-1\leq u_k \leq 1
\end{align*}%}
%\eva{More details about this + citation}


For this example, $\Delta$ and $\varepsilon_Q$ are fixed to $0.1$ and $0.001$,
respectively. By changing the time horizon $N$, we evaluate the scalability
of our approach for controllers with very large number of regions, as 
increasing $N$ generally results in a higher number of regions for the output controller. To
compute the explicit model predictive controllers for different time horizons,
we use Matlab's MPC toolbox. \todo{is this the same as before? If so, add a 'again'}
 
%For this example, we fix the values for $\delta$ to $0.1$ and $\varepsilon$ to $0.001$,
%but we consider also the prediction horizon $N$, as an input parameter for the
%analysis. At design time, the prediction horizon of the controller m models how
%many future iterations of the feed-forward system are considered in the
%optimization problem (infinite horizon would correspond to the optimal
%controller). Note that in this example, the number of regions is highly
%correlated with the prediction horizon m of the controller: the number of regions
%increases with larger m, and thus also increases the execution time of our analysis.
%Even if this is a general behavior, it is not the rule.
%\eva{What does the above sentence mean?}

The design of the controller in Matlab takes a few minutes, then controllers and
hyperplanes are encoded in Daisy for finite-precision assignment. On average, the
analysis of a single controller or hyperplane in Daisy requires less than two
minutes and less than 500MB of memory. The finite-precision assignment is trivially parallelizable, 
because any controller or hyperplane can be analyzed independently.
We thus run this analysis on a cluster with 48 cores, and note that the 
memory consumption remains below 15GB. 
%\eva{There should be an explanation why this is not providing end-to-end robustness}


\autoref{tab:di} shows the results of this experiment.
We observe that our analysis scales up to 1829 controllers and 14632 hyperplanes in a few hours.  The computed minimal uniform precision saves on average $49.8\%$ of memory
compared to a uniform $32$ bit baseline overall (for F, G, H and K together). 
Moreover, mixed-precision reduces the memory requirement by an additional
$9.6\%$ on average with respect to the minimal uniform baseline.
We further observe that relative savings remain largely constant for different 
prediction horizons.
% \autoref{tab:di} shows the stability of memory tuning in Daisy is
% constant with respect to the number of controllers or hyperplanes. This is true
% for both F, G and H, K. Indeed, the memory gain from uniform in 32bits and
% minimal uniform (\%32vU) is constant among different input values m. The same
% holds for (\%UvM).
\input{Conclusion}



\bibliographystyle{ACM-Reference-Format}
\bibliography{biblio}

\end{document}
