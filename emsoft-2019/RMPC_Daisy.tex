% !TEX root = main.tex
\section{Explicit MPC and Finite Precision}
\subsection{Robust Explicit MPC}
\def\reals{\mathbb{R}}
We consider the class of linear time-invariant (LTI) systems characterized by the difference equation
\begin{equation}
\label{eq:DSS}
x_{k+1}=Ax_k+Bu_k+Ew_k,\quad  k=0,1,2,\ldots
\end{equation}
where $x\in \reals^{n\times 1}$ is the state, $u\in \reals^{m\times 1}$ is the control input and $w\in \reals^{d\times 1}$ is the disturbance. Matrices $A\in \reals^{n\times n}$, $B\in \reals^{n\times m}$ and $E\in \reals^{n\times d}$ are capturing respectively the effects of current state, input and disturbance on the next state. We assume the disturbance $w$ belongs to a set $\mathcal{W}$ where $\mathcal{W}$ is a polyhedral set.
In this paper, we focus on robust formulation of MPC which at each time step, minimizes the worst-case value of the objective function with respect to the disturbances over the control inputs. We also consider the objective function to be quadratic with respect to the states and inputs. To reduce the the conservativeness of the optimization and enlarge the set of feasible input trajectories, we assume the pair $(A,B)$ is stablizable and
$u_k=Kx_k+v_k$, where $K$ is selected such that some desired property such as stability is satisfied, and synthesize $v_k$ for the system.
%The modifed state evolution of the system will be $x_{k+1}=\bar Ax_k+Bv_k+Ew_k$ with $\bar A := A +BK$.
% and $v_k=\kappa(x_k)$ denotes the output of model predictive controller.  
The constrained optimization at each time step is of the form
\begin{align}
\label{eq:RMPC_prob}
J^{\ast}(x_0)=\min_{v_0,\cdots,v_{N-1}}& \max_{w_0,\cdots,w_{N-1}} \sum_{i=0}^{N-1}(x_i^TQx_i+u_i^TRu_i) + x_N^TQ_Fx_N\nonumber\\
\text{s.t.} \quad &x_{i+1}=Ax_i+Bu_i + E w_i, \quad\forall i\in\{0,1,\cdots,N-1\}\nonumber\\
\quad &u_i=Kx_i+v_i, \quad\forall i\in\{0,1,\cdots,N-1\}\nonumber\\
&u_i\in\mathcal{U},\quad \forall w_i\in\mathcal{W}, \quad\forall i\in\{0,1,\cdots,N-1\}\nonumber\\
&x_i\in\mathcal{X},\quad \forall w_i\in\mathcal{W},\quad\forall i\in\{0,1,\cdots,N\},
\end{align}
%\Sadegh{The role of disturbance is missing in the inequalities. We should say, we want to satisfy the inequalities for all possible values of the disturbance.}
%
where $\mathcal U$ and $\mathcal X$ are polyhedral sets as feasible sets of inputs and states. Matrices $Q\in\reals^{n\times n}$ and $Q_F\in\reals^{n\times n}$ are positive definite, $R\in\reals^{m\times m}$ is positive semidefinite and $N$ denotes the length of the prediction horizon.
\begin{theorem}[\cite{delaPea:2005}]
The optimization  \eqref{eq:RMPC_prob} can be translated into a multi-parametric quadratic problem which admits a closed-form solution. Furthermore, for the case that $R>0$, the controller is $u_k = K x_k + \kappa(x_k)$ with $\kappa(\cdot)$ being a continuous piecewise affine (PWA) function over polyhedral regions:
\begin{equation}
\kappa(x_k)=
\begin{cases}
F_1x_k+G_1 & \text{if $x_k\in \mathcal{R}_1$}\\
F_2x_k+G_2 & \text{if $x_k\in \mathcal{R}_2$}\\
\vdots\\
F_Px_k+G_P & \text{if $x_k\in \mathcal{R}_P$}
\end{cases} 
\end{equation}
where $\mathcal{R}_i$ is a ploytope determined by a set of linear inequalities $\mathcal R_i = \{x\in\mathcal X\,|\,H_ix\leq K_i\}$. 
\end{theorem}
The essential idea behind this theorem is to utilize the closed form $x_i=A^ix_0+\sum_{l=0}^{i-1}A^{i-l-1}Bu_l$ and transform the optimization \eqref{eq:RMPC_prob} into the following quadratic program:
%\begin{align}
%	&\min_{U_N}U_N^TZU_N+x_0^TVU_N+x_0^TYx_0\nonumber\\
%	&s.t.\quad JU_N\leq s+Dx_0
%\end{align} 
\begin{align}
\label{eq:multi_param_prog}
J^{\ast}(x_0)=\min_{z,\gamma}& \left[x_0^TYx_0+\frac{1}{2}z^THz+\gamma\right]\\
\text{s.t.} \quad &G_mz+g_m\gamma\leq W_m+S_mx_0,\nonumber\\
&G_cz\leq W_c+S_cx_0\nonumber
\end{align}
where, $z\in \reals^{mN}$ and $\gamma\in\reals$ are decision variables and $Y$, $H$, $G_m$, $G_c$,$S_m$, $S_c$, $W_m$, $W_c$, $g_m$ are matrices of proper sizes that can be easily obtained from the original optimization \eqref{eq:RMPC_prob}  (see \cite{delaPea:2005}).

%\Sadegh{Adding a remark and saying that the approach works as long as the solution of formulated optimization has a piecewise affine form?} 

The optimization \eqref{eq:multi_param_prog} can be solved using multi-parametric techniques to compute the explicit form of $\kappa(\cdot)$.
Efficient implementation of such computations is available in multi-parametric toolbox of MATLAB~\cite{matlabMPT, matlabYALMIP}.
Let us denote the set of states $x_0$ for which the optimization  \eqref{eq:multi_param_prog} is feasible by $\mathcal X_s$. Then $\mathcal X_s\subseteq \mathcal X$ and is is the union of all regions $\mathcal{R}_{i}$:
\begin{equation}
\mathcal X_s = \bigcup_{i}(H_{i}\statevar<=K_{i}).
\end{equation}


The actual implementation of the controller requires storing the matrices $F_i,G_i,H_i,K_i$ for all $i\in\{1,2,\ldots,P\}$. At each time step $k$, the current state $x_k$ is used to detect the right polytope $i$ such that $H_i x_k\le K_i$. Then the control action $v_k = F_i x_k + G_i$ is computed and applied to the system.

%Therefore, given the state vector $x$, to compute control input, one has to first determine to which polytope does the state vector belong.
Several algorithms are proposed in the literature \cite{Mnnigmann:2011,Jones:2006} to find the right polytope $i$ to which the state $x_k$ belong. The most straightforward technique is a linear search over all polytopes in \statespace\space using a switch/case conditional statement of Listing \ref{lst:caseof}.
%\Sadegh{Perhaps we can mention another approach here? add references.}

\begin{lstlisting}[escapeinside={(*}{*)},label={lst:caseof}, caption=switch for region selection]
switch((*\statevarmath*)):
case (*$ H_{1}\statevar<=K_{1}$*) then (*$v_{k}=F_1x_k+G_1 $*)
case (*$ H_{2}\statevar<=K_{2}$*) then (*$v_{k}=F_2x_k+G_2$*)
case (*$ H_{3}\statevar<=K_{3}$*) then (*$v_{k}=F_3x_k+G_3$*)
...
case (*$ H_{P}\statevar<=K_{P}$*) then (*$v_{k}=F_Px_k+G_P$*)
\end{lstlisting}
%Once the polytope was determined, computing for $F_ix+G_i$ requires even less effort as it only involves summation and multiplication.  

To keep the costs of implementation down, low-end microcontrollers usually have limited memory and computational power. Therefore, implementation of the above explicit MPC on microcontrollers will bring some issues since all regions' matrices ($H_i$s and $K_i$s) and activation functions ($F_i$s and $G_i$s) have to be stored on the microcontroller. The goal of our paper is present an approach for minimizing the memory usage of the implementation while satisfying the hard constraints on the states. 
%\section{Finite Precision Implementation}
%\Sadegh{To be included by Rocco}