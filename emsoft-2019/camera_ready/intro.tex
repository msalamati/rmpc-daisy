% !TEX root = main.tex
\section{Introduction}
%\eva{General comment: should we add some more references to related work (and not only to MPC)?}

Model predictive control (MPC) is a technique to design control actions by solving finite-horizon open-loop
optimal control problems at each sampling instant \cite{RMD17}.
The result of each optimization gives a sequence of optimal control actions, only the first of which is applied
to the process.
The same procedure is applied in the next time instant with a shifted time horizon and a new initial state, 
after receiving the updated values of the process state.
The optimization problem in MPC uses a dynamic model of the process, encodes all input and output (state) constraints, and 
optimizes a performance index. 
MPC has shown to be successful in a wide variety of industrial applications \cite{QinBadgwell03}, due to its 
ability to systematically handle processes with many state and input variables as well as constraints on them. 

The main difference between MPC and conventional control is in the nature of the function that maps the measured outputs to control actions. 
MPC computes such a function \emph{online}, whereas a conventional controller pre-computes the function offline.
The online computations required in MPC limits its applicability to slow processes and fast computation platforms: 
the sampling time has to be large enough and the platform fast enough to allow enough time for solving the optimization problem
and obtaining the optimal action for the next time instance. 
Moreover, the optimization solver needs to be certified when using MPC in safety critical applications \Mahmoud{\cite{Alessio2009}}.

One way to tackle these problems is through \emph{explicit MPC} (EMPC) \cite{Bemporad:2002,Alessio2009}, which formulates the optimization
problem but computes offline a symbolic representation of the solution as a function of the state.
At run-time, the solution is evaluated on the current state as in conventional control.
For example, for linear time invariant models with linear constraints and quadratic costs, the optimization
problem for MPC can be modeled as a quadratic program, and EMPC techniques solve the optimization problem
using multi-parametric programming techniques.
The explicit solution is representable as a partition of the controller domain into a number of polyhedral regions,
and an affine map for each region. 
\Mahmoud{Given the current state, EMPC draws the corresponding affine mapping out of a stored lookup table and evaluates the control.}
Explicit MPC thus expands the class of systems being controlled by MPC strategies, by taking out the need to \Mahmoud{perform massive online computations}
or to certify a complex optimization routine \Mahmoud{\cite{memoryMPC}}.

However, there are still bottlenecks in implementing EMPC on resource-constrained embedded micro-controllers.
First, implementations of EMPCs \Mahmoud{over industrial micro-controllers with limited memory} can suffer from large memory usage, because the solution of the optimization
problem can involve many (often hundreds) of regions~\cite{memoryMPC}.
Since the memory consumption grows linearly with the number of regions, 
this can be a limiting factor in using EMPC on resource-constrained micro-controllers.
Second, many low-end micro-controllers only support fixed-point arithmetic.
Errors in the controller implementation are inversely proportional to the number of bits used for representing each variable~\cite{Anta10}. 
An implementation of EMPC has to be \emph{robust} to 
implementation errors to be able to enforce hard constraints on the states at run time~\cite{imperialrmpc}.

In this paper, we consider the problem of implementing EMPC on low-end microcontrollers with fixed-point arithmetic
in a \emph{memory-efficient} and \emph{robust} way.
We propose an automatic controller and mixed-precision implementation co-design
technique that computes mixed-precision assignments which minimize bitwidths for all
variables while ensuring the resulting implementation error remains within the
robustness margin.


Our proposed method iteratively solves a robust version of MPC 
\Mahmoud{that} considers finite-precision implementation errors as disturbances
in the dynamics \cite{RamirezC06}; see~\autoref{fig:overview}.
Initially, we estimate a bound $\Delta = \Delta_0$ on the implementation error and solve a
min-max quadratic program for explicit robust MPC \cite{delaPea:2005,RamirezC06}, where the system
model has an explicit disturbance bounded by $\Delta$.
The solution is represented as a set of polyhedral regions, and an affine map for each region.

Next, we consider the error in the control input arising out of a fixed-point implementation of the solution.
The error has two sources.
First, the fixed point implementation may pick a different polyhedral region (due to imprecision
in checking membership in a polyhedron).
Second, the fixed-precision implementation of the affine function will have a numerical error due to quantization.
We bound both sources of error statically and automatically,  by finding the maximum possible error due to incorrect region selection
as well as a static bound on the error in computing the affine map.
If the error is at most $\Delta$, we know that the implementation satisfies the robustness margin in the model and
we use an automated mixed-precision tuning tool to find an optimal bitwidth allocation that keeps the implementation below the error bound.
If not, we increase the error bound $\Delta$ and run the loop again to find a new robust controller and find its best mixed-precision
implementation. %\Mahmoud{While our formulation is general enough to take the effect of measurement/process noise into account, this is not subject of study of the current paper.}

To the best of our knowledge, our algorithm is the first to automatically and soundly
synthesize a suitable fixed-point precision for explicit robust MPC controllers,
and the first to consider mixed-precision in this context.

We have implemented our algorithm on top of Matlab's multi-parametric toolbox 
for robust explicit MPC~\cite{matlabMPT}
and the Daisy tool for multi-precision tuning and fixed-point error analysis~\cite{Daisy}. %\Mahmoud{It should be noticed that MATLAB and Daisy are solely used for offline computations. The output of our algorithm contains the information required for implementation of the verified controller over embedded systems.}

We have applied our technique on a number of standard benchmark examples. \Mahmoud{Taking all these benchmarks into account,}
our algorithm finds mixed-precision implementations \Mahmoud{that} save up to 20.9\%
memory (on average, 12.6\%) in controller implementations over a minimal
uniform-precision implementation (and an average saving of 46.8\% over a uniform
32-bit implementation), while maintaining the correctness of the controller.

For smaller examples, the analysis takes only a few minutes of computation.
We also demonstrate the scalability of our mixed-precision tuning and error bounds analysis:
we show that on explicit MPC controllers with thousands of regions, our tool finishes in a few hours.
In absolute terms, the memory saving corresponds to 21KB over a uniform
precision implementation on our largest benchmark. %\Mahmoud{It should be noticed that these savings are not affecting the run-time. Synthesis of the controller and the corresponding precisions are all performed offline.}

In summary, our contributions are:
\begin{itemize}
\item We provide an end-to-end automatic tuning algorithm for robust explicit MPC that ensures \Mahmoud{that}
hard constraints on the state are met despite numerical errors;
\item We design a static and scalable error analysis algorithm for piecewise affine functions to bound
errors in controller implementations; and
\item Through a set of standard control benchmarks, we demonstrate that our tuning algorithm
can achieve up to 20\% savings in memory while maintaining robustness.
\end{itemize} 

