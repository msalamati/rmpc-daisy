% !TEX root = main.tex
\section{Background}
In this section, we provide relevant background on robust explicit MPC,
fixed-point arithmetic and error analysis.

\subsection{Robust Explicit MPC}
\def\reals{\mathbb{R}}
We consider the class of linear time-invariant (LTI) systems characterized by the difference equation
\begin{equation}
\label{eq:DSS}
x_{k+1}=Ax_k+Bu_k+Ew_k,\quad  k=0,1,2,\ldots
\end{equation}
where $x\in \reals^{n\times 1}$ is the state, $u\in \reals^{m\times 1}$ is the control input and $w\in \reals^{d\times 1}$ is the disturbance. Matrices $A\in \reals^{n\times n}$, $B\in \reals^{n\times m}$ and $E\in \reals^{n\times d}$  capture respectively the effects of current state, input and disturbance on the next state. We assume the disturbance $w$ belongs to a set $\Mahmoud{\mathcal{W}_{\Delta}}$ where $\Mahmoud{\mathcal{W}_{\Delta}}$ is a polyhedral set \Mahmoud{defined by \autoref{eq:W_delta}}.
In this paper, we focus on the robust formulation of MPC which at each time step
minimizes the worst-case value of an objective function with respect to the
disturbances over the control inputs. We assume the objective function is
quadratic with respect to the states and inputs.
We assume a linear translation on the input with the form
$$u_k=\mu x_k+v_k$$
and synthesize $v_k$ instead of $u_k$. This choice reduces the 
conservativeness of the optimization and enlarges the set of feasible input
trajectories. The matrix $\mu$ is selected such that some desired property is satisfied under suitable assumptions on the system, e.g., stability if the pair $(A,B)$ is stabilizable.

%\eva{I don't understand the above sentence. Maybe split into several sentences? Where does $u_k$ come from (the sentence right now says you assume $u_k$)?}
%The modifed state evolution of the system will be $x_{k+1}=\bar Ax_k+Bv_k+Ew_k$ with $\bar A := A +BK$.
% and $v_k=\kappa(x_k)$ denotes the output of model predictive controller.  
The constrained optimization at each time step is of the form
\begin{align}
\label{eq:RMPC_prob}
J^{\ast}(x_0)=&\min_{v_0,\cdots,v_{N-1}} \max_{w_0,\cdots,w_{N-1}} \sum_{i=0}^{N-1}(x_i^TQx_i+u_i^TRu_i) + x_N^TQ_Fx_N\nonumber\\
\text{s.t.} \quad &x_{i+1}=Ax_i+Bu_i + E w_i, \quad\forall i\in\{0,1,\cdots,N-1\}\nonumber\\
\quad &u_i=\mu x_i+v_i, \quad\forall i\in\{0,1,\cdots,N-1\}\nonumber\\
&u_i\in\mathcal{U},x_i\in\mathcal{X},\quad \forall w_i\in\Mahmoud{\mathcal{W}_{\Delta}},\,\,\forall i\in\{0,1,\cdots,N\},
%&x_i\in\mathcal{X},\quad \forall w_i\in\mathcal{W},\quad\forall i\in\{0,1,\cdots,N\},
\end{align}
%\Sadegh{The role of disturbance is missing in the inequalities. We should say, we want to satisfy the inequalities for all possible values of the disturbance.}
%
where $\mathcal U$ and $\mathcal X$ are polyhedral sets denoting the feasible
sets of inputs and states. Positive definite matrices $Q\in\reals^{n\times n}$ and
$Q_F\in\reals^{n\times n}$ indicate weights on the states. $R\in\reals^{m\times m}$ is
positive semidefinite and indicates a weight on the input in the objective function. $N$ denotes the length of the prediction horizon.
%\eva{What are $Q, Q_F, R$ and where do they come from?}

\begin{theorem}[\cite{delaPea:2005}]
\label{thm:EMPC}
The optimization \eqref{eq:RMPC_prob} can be translated into a multi-parametric quadratic problem which admits a closed-form solution. Furthermore, for the case that $R>0$, the controller is $u_k = \mu x_k + \kappa(x_k)$ with $\kappa(\cdot)$ being a continuous piecewise affine (PWA) function over polyhedral regions:
\begin{equation}
\label{eq:affine_map}
\kappa(x_k)=
\begin{cases}
F_1x_k+G_1 & \text{if $x_k\in \mathcal{R}_1$}\\
F_2x_k+G_2 & \text{if $x_k\in \mathcal{R}_2$}\\
\vdots\\
F_Px_k+G_P & \text{if $x_k\in \mathcal{R}_P$}
\end{cases} 
\end{equation}
where $\mathcal{R}_i$ is a polyhedral region determined by a set of linear inequalities $\mathcal R_i = \{x\in\mathcal X\,|\,H_ix\leq K_i\}$. 
\end{theorem}
The essential idea behind this theorem is to utilize the closed form $x_i=A^ix_0+\sum_{l=0}^{i-1}A^{i-l-1}Bu_l$ and transform the optimization \eqref{eq:RMPC_prob} into the following quadratic program:
%\begin{align}
%	&\min_{U_N}U_N^TZU_N+x_0^TVU_N+x_0^TYx_0\nonumber\\
%	&s.t.\quad JU_N\leq s+Dx_0
%\end{align} 
\begin{align}
\label{eq:multi_param_prog}
J^{\ast}(x_0)=\min_{z,\gamma}& \left[x_0^TYx_0+\frac{1}{2}z^THz+\gamma\right]\\
\text{s.t.} \quad &G_mz+g_m\gamma\leq W_m+S_mx_0,\nonumber\\
&G_cz\leq W_c+S_cx_0\nonumber
\end{align}
where $z\in \reals^{mN}$ and $\gamma\in\reals$ are decision variables, and $Y$, $H$, $G_m$, $G_c$, $S_m$, $S_c$, $W_m$, $W_c$, and $g_m$ 
are matrices of proper sizes that can be easily obtained from the original optimization \eqref{eq:RMPC_prob}  (see \cite{delaPea:2005}).

%\Sadegh{Adding a remark and saying that the approach works as long as the solution of formulated optimization has a piecewise affine form?} 

The optimization \eqref{eq:multi_param_prog} can be solved using multi-parametric techniques to compute the explicit form of $\kappa(\cdot)$.
An efficient implementation of such computations is available in the multi-parametric toolbox of Matlab~\cite{matlabMPT, matlabYALMIP}.
Let us denote the set of states $x_0$ for which the optimization  \eqref{eq:multi_param_prog} is feasible by $\mathcal X_s$. Then $\mathcal X_s\subseteq \mathcal X$ and is the union of all regions $\mathcal{R}_{i}$:
\begin{equation}
\mathcal X_s = \bigcup_{i}(H_{i}\statevar\leq K_{i}).
\end{equation}


The implementation of the controller stores the matrices $F_i,G_i,H_i,K_i$ for all $i\in\{1,2,\ldots,P\}$. 
At each time step $k$, the current state $x_k$ is used to detect the polyhedral region 
such that $H_i x_k\le K_i$. Then the control action $v_k = F_i x_k + G_i$ is computed and applied to the system.
Several algorithms are proposed in the literature \cite{Mnnigmann:2011,Jones:2006} to find the right polyhedral region $i$ to which the state $x_k$ belong. 
The most straightforward technique is a linear search over all polyhedral regions (Fig.~\ref{lst:caseof}).

\begin{figure}[t]
\begin{lstlisting}[mathescape=true,basicstyle=\small\ttfamily,morekeywords={if, then, elseif, return}]
if ($H_{1}\statevar \leq K_{1}$) then $v_{k}=F_1x_k+G_1$
elseif ($H_{2}\statevar \leq K_{2}$) then $v_{k}=F_2x_k+G_2$
elseif ($H_{3}\statevar \leq K_{3}$) then $v_{k}=F_3x_k+G_3$
...
elseif ($ H_{P}\statevar\leq K_{P}$) then $v_{k}=F_Px_k+G_P$
return $v_k$
\end{lstlisting}
\caption{Structure of an explicit MPC controller}
\label{lst:caseof}
\end{figure}
%Once the polytope was determined, computing for $F_ix+G_i$ requires even less effort as it only involves summation and multiplication.  

To keep the implementation cost down, low-end microcontrollers usually have
limited memory and computational power. Therefore, the implementation of the above
explicit MPC on microcontrollers will bring some issues since all regions'
matrices ($H_i$s and $K_i$s) and activation functions ($F_i$s and $G_i$s) have
to be stored on the microcontroller. Our paper presents an
approach for minimizing the memory usage of the implementation while satisfying
the hard constraints on the states.
%\section{Finite Precision Implementation}
%\Sadegh{To be included by Rocco}
