%!TEX root = main.tex
\section{Experimental Results}\label{sec:experiments}
\begin{table*}[p]
  \centering
  \caption{Inverted Pendulum and Aircraft.
  \textmd{ 
  Memory requirements in number of bits for storing F, G and H, K for uniform 32
  bit precision (Uni32), uniform custom precision (Uni, word length chosen in
  parentheses) and mixed precision (Mix), for different values of $\Delta$ and
  $\varepsilon_Q$.
  The error due to choosing a wrong region is denoted as `max' and the finite-precision
  error bound in evaluating the control actions as $err(u_{i})$.
  \%32vsU is the percentage of memory saved using Uni compared to the baseline Uni32,
  and \%UvsM is the percentage of memory saved using Mix compared to Uni.
  % delta is the disturbance used to design the controller, and $\varepsilon$ the
  % size of the safe space between two generic regions (\texttt{tube}). We fix
  % $\varepsilon$ and try different delta (in red) and vice-versa. The maximal
  % error due to a wrong activation function is \maxUij and the error bound for
  % finite precision activation functions $err(u_{i})$. F, G and H, K represent
  % the memory requirements for activation functions, and hyperplanes. Uni32 is
  % the total number of bits for the controller using 32 bits precision. Uni is
  % the uniform precision found by our analysis, together with the format. Mix is
  % the total number of bits for the controller in mixed-precision. \%32vsU is the
  % difference between the baseline Uni32 and Uni (in percentage), then \%UvsM is
  % the difference between Uni and Mix.
  }}
  \label{tab:ipd}
  \renewcommand{\arraystretch}{1.2}
  \setlength{\tabcolsep}{0.7em} % for the horizontal padding
  \begin{tabular}{l|rr|rrrrr|rrrrr}
    \toprule
     & & & \multicolumn{5}{c}{F and G} & \multicolumn{5}{|c}{H and K} \\
    %\cline{2-13}
    %\cline{5-6}
    %\cline{9-12}
    %\cline{13-16}
    
    \multirow{14}{*}{\rotatebox{90}{pendulum}} &  
    \multicolumn{1}{c}{$\Delta$}&
    \multicolumn{1}{c}{$\varepsilon_Q$} &
    %\multicolumn{1}{c}{max} &
    %\multicolumn{1}{c}{$err(u_{i})$} &
    \multicolumn{1}{|c}{Uni32}&
    \multicolumn{1}{c}{Uni}&
    \multicolumn{1}{c}{Mix}&
    \multicolumn{1}{c}{\%32vU}&
    \multicolumn{1}{c}{\%UvM}&
    \multicolumn{1}{|c}{Uni32}&
    \multicolumn{1}{c}{Uni}&
    \multicolumn{1}{c}{Mix}&
    \multicolumn{1}{c}{\%32vU}&
    \multicolumn{1}{c}{\%UvM} \\
    
    \midrule

    & \textbf{0.30} & 0.001   & 2688 & 924 (p=11) & 759 & 65.6\% & 17.9\% & 11136 & 5568 (p=16) & 4991 & 50.0\% & 10.4\% \\
    
    & \textbf{0.20} & 0.001   & 2688 & 924 (p=11) & 806 & 65.6\% & 12.8\% & 11136 & 5568 (p=16)& 4992 & 50.0\% & 10.3\% \\
    
    & \textbf{0.10} & 0.001   & 2688 & 1008 (p=12) & 908 & 62.5\% & 9.9\% & 11136 & 5568 (p=16)& 4992 & 50.0\% & 10.3\% \\
    
    & \textbf{0.08} & 0.001   & 2688 & 1092 (p=13) & 946 & 59.4\% & 13.4\% & 11136 & 5568 (p=16) & 4992 & 50.0\% & 10.3\% \\
    
    & \textbf{0.05} & 0.001   & 2688 & 1176 (p=14) & 1030 & 56.3\% & 12.4\% & 11136 & 5568 (p=16) & 4992 & 50.0\% & 10.3\% \\ 
    
    % Empy Line
    & & & & & & & & & & & &\\
    % Empy Line
    
    & 0.1 & \textbf{0.0006}  & 2688 & 1008 (p=12) & 891 & 62.5\% & 11.6\% & 11136 & 5916 (p=17) & 5261 & 46.9\% & 11.1\% \\
    
    & 0.1 & \textbf{0.0008}  & 2688 & 1008 (p=12) & 901 & 62.5\% & 10.6\% & 11136 & 5568 (p=16) & 5135 & 50.0\% & 7.8\% \\
    
    & 0.1 & \textbf{0.0010}  & 2688 & 1008 (p=12) & 908 & 62.5\% & 9.9\% & 11136 & 5568 (p=16) & 4992 & 50.0\% & 10.3\% \\
    
    & 0.1 & \textbf{0.0030}  & 2688 &  1092 (p=13) & 993 & 59.4\% & 9.1\% & 11136 & 4872 (p=14) & 4462 & 56.3\% & 8.4\% \\
    
    & 0.1 & \textbf{0.0050}  & 2688 & 1680 (p=20) & 1527 & 37.5\% & 9.1\% & 11136 & 4872 (p=14) & 4204 & 56.3\% & 13.7\% \\
    
    \midrule

    \multirow{12}{*}{\rotatebox{90}{aircraft}}
    & \textbf{0.30} & 0.001  & 9984 & 6864 (p=22) & 6210 & 31.3\% & 9.5\% & 79872 & 64896 (p=26) & 53059 & 18.8\% & 18.2\% \\
    
    & \textbf{0.20} & 0.001  & 10368 & 7452 (p=23) & 6725 & 28.1\% & 9.8\% & 82944 & 67392 (p=26) & 55098 & 18.8\% & 18.2\% \\
    
    & \textbf{0.10} & 0.001  & 10368 & 7776 (p=24) & 7134 & 25.0\% & 8.3\% & 82944 & 67392 (p=26) & 55098 & 18.8\% & 18.2\% \\
    
    & \textbf{0.08} & 0.001  & 10368 & 7776 (p=24) & 7275 & 25.0\% & 6.4\% & 82944 & 67392 (p=26) & 55098 & 18.8\% & 18.2\% \\
    
    & \textbf{0.05} & 0.001  & 10368 & 8424 (p=26) & 7840 & 18.8\% & 6.9\% & 82944 & 67392 (p=26) & 55098 & 18.8\% & 18.2\% \\
    
    % Empy Line
    & & & & & & & & & & & &\\
    % Empy Line
    
    & 0.1 & \textbf{0.0006}  & 10368 & 7776 (p=24) & 7047 & 25.0\% & 9.4\% & 82944 & 69984 (p=27) & 55705 & 15.6\% & 20.4\% \\
    
    & 0.1 & \textbf{0.0008}  & 10368 & 7776 (p=24) & 7125 & 25.0\% & 8.4\% & 82944 & 69984 (p=27) & 57859 & 15.6\% & 17.3\% \\
    
    & 0.1 & \textbf{0.0010}  & 10368 & 7776 (p=24) & 7134 & 25.0\% & 8.3\% & 82944 & 67392 (p=26) & 55098 & 18.8\% & 18.2\% \\
    
    & 0.112* & \textbf{0.0030}  & 10368 &  9720 (p=30) & 9051 & 6.3\% & 6.9\% & 82944 & 64800 (p=25) & 52754 & 21.9\% & 18.6\% \\
    
    & 0.185* &\textbf{0.0050}  & 10368 & 9720 (p=30) & 9051 & 6.3\% & 6.9\% & 82944 & 62208 (p=24) & 47877 & 25.0\%& 23.0\% \\
    \bottomrule
  \end{tabular}
\end{table*}
\begin{table*}
  \centering
  \caption{Double Integrator.\textmd{ $N$ is the prediction horizon in RMPC, time gives the execution time in minutes, Regs is the number of regions of the controller with Hyps hyperplanes. Uni32 is the total number of bits when all operations are in 32 bits, Uni the minimal uniform precision required, Mix is mixed-precision, \%32vU and \%UvM give the improvements of uniform and mixed precisions.}}
  \label{tab:di}
  \renewcommand{\arraystretch}{.2}
  \setlength{\tabcolsep}{-0em} % for the horizontal padding
  \begin{tabular}{lccc|lcccl|lcccl}
    \toprule
    \multicolumn{4}{c}{} & \multicolumn{5}{|c|}{$F$ and $G$} & \multicolumn{5}{c}{$H$ and $K$} \\
    %\hline
    %\cline{5-6}
    %\cline{9-12}
    %\cline{13-16}
    \multicolumn{1}{c}{$N$\:}&
    \multicolumn{1}{c}{time\:}&
    %\multicolumn{1}{c}{Eps} &
    \multicolumn{1}{c}{Regs} &
    \multicolumn{1}{c}{Hyps} &
    %\multicolumn{1}{c}{\maxUij} &
    %\multicolumn{1}{c}{Err-Bound} &
    \multicolumn{1}{|c}{Uni32}&
    \multicolumn{1}{c}{Uni}&
    \multicolumn{1}{c}{Mix}&
    \multicolumn{1}{c}{\%32vU\;}&
    \multicolumn{1}{c}{\%UvM}&
    \multicolumn{1}{|c}{Uni32}&
    \multicolumn{1}{c}{Uni}&
    \multicolumn{1}{c}{Mix}&
    \multicolumn{1}{c}{\%32vU\;}&
    \multicolumn{1}{c}{\%UvM} \\
    \midrule
    2 & 2 & 9 & 72 & 1728 & 810 (p=15) & 628 & 53.1\% & 22.5\% & 13824 & 7776 (p=18) & 7280 & 43.8\%& 6.4\% \\
    5 & 9 & 53 & 424 & 10176 & 5088 (p=16) & 3623 & 50.0\% & 28.8\% & 81408 & 45792 (p=18) & 42656 & 43.8\% & 6.8\% \\
    8 & 23 & 143 & 1144 & 27456 & 13728 (p=16) & 9864 & 50.0\%  & 28.1\% & 219648 & 123552 (p=18) & 114948 & 43.8\% & 7.0\% \\
    11 & 47 & 277 & 2216 & 53184 & 26592 (p=16) & 18980 & 50.0\% & 28.6\% & 425472 & 239328 (p=18) & 222616 & 43.8\% & 7.0\% \\
    
    14 & 73 & 431& 3446& 82752& 41376 (p=16)& 28685& 50.0\% & 30.7\% & 661632& 372168 (p=18)& 346020& 43.8\% &7.0\% \\
    
    17 & 106 & 621 & 4968 & 119232 & 59616 (p=16) & 40503 & 50.0\% & 32.1\% & 953856& 536544 (p=18)& 498668& 43.8\% & 7.1\% \\
    
    21 & 150 & 928 & 7432 & 178368 & 89184 (p=16) & 59409 & 50.0\% & 33.4\% & 1426944 & 802656 (p=18) & 745936 & 43.8\% & 7.1\%  \\
    
    25 & 223 & 1299 & 10392 & 249408 & 124704 (p=16) & 81889 & 50.0\%& 34.3\%& 1995264 & 1122336 (p=18)& 1043456 & 43.8\%& 7.0\% \\
    
    40 & 314 & 1829& 14632\;& 351168\:& 175584 (p=16)& 113979& 50.0\%& 35.1\%& 2809344\;& 1580256 (p=18)& 1469834& 43.8\%& 7.0\% \\
    
    %25 & 445 &1299 & 23382 & 249408 & 109116 (p=14) & 80524 & 56.3\% & 26.2\% & 2992896 & 1589976 (p=17) & 1355675 & 46.9\% & 14.7\% \\
    \bottomrule
  \end{tabular}
\end{table*}

We evaluate a prototype implementation of the algorithm in~\autoref{lst:alg}
on three examples. For the first two, we apply the complete pipeline (design and memory optimization)
which returns an end-to-end robust controller. With the third example, we evaluate
the scalability of our approach when the number of regions and hyperplanes are in
the order of tens of thousands.

% This section reports experiments with the prototype described
% in~\autoref{lst:alg} applied to three different benchmarks: in the former two,
% we evaluate the complete pipeline (design and memory optimization) to produce in
% output a complete controller. In the last experiment, we want to show the
% scalability of our approach when the number of regions and hyperplanes are in
% the order of tens of thousands.

% For the design of controllers, we rely on MATLAB MPC toolbox. The outputs of the
% suite are vectors F, G and H, K for activation functions and hyperplanes
% equations. We encode both of them in Daisy for finite-precision optimization.
The design of end-to-end robust controllers has been performed on a laptop with Intel i7-6700HQ CPU at
2.60GHz, with 16GB of RAM. The evaluation of the last benchmark runs on a
cluster with 48 Intel Xeon v2 @ 3.00GHz cores with 1TB of RAM, of which our analysis
only used 15GB.
%Note that the memory consumption of our analysis never exceed 15GB of memory.

\subsection{End-to-End Robust Controller}
%\eva{Use the same symbol for Delta and Eps as used in the technical section.}

We evaluate our complete pipeline on two benchmarks. 
The first one is the inverted pendulum problem depicted in Section \ref{sec:example}, where we set
$m=0.344$ kg, $b=0.48$ N s/m, $L=1.703$ m and $T_s=0.1$ s. The gain $\mu$ is
selected such the $\bar A$ has poles at $-0.1$ and $-0.5$. Moreover, we select
$N=2$, $R=1$ and $Q=Q_F=100\mathbf{\it I}$, where $\mathbf{ \it I}$ denotes the
identity matrix of proper size. 

Our second benchmark is a well-known 4D example for aircraft
controller design~\cite{Kapasouris:1998}.
The control inputs for the aircraft 4-D model are the elevator and flaperon angles,
and the attack and pitch angles are the output states that need to be regulated.
The open-loop system is unstable as it has a pole with positive real part. Both
control inputs are constrained between $\pm25$ degrees. The outputs are
only constrained during the first prediction horizon. You also specify scale
factors for outputs. Using the gain $\mu$, the poles of $\bar A$ are placed at
$-5$, $-3$, $-1$ and $-2$. The robust MPC problem is solved for $N=2$,
$R=\mathbf{\it I}$ and $Q=Q_F=5\mathbf{\it I}$. Note that for both of the
examples, matrices $R$, $Q$ and $Q_F$ are selected such that convergence to the
origin is given more weight compared to the control effort as long as
constraints are satisfied.


%the feasible domain $X$ is bounded in the range $X_{0} \in
%(-3.23, 3.23)$ and $X_{1}\in (-1.46, 1.46)$. MATLAB designs a controller with 14
%regions and 58 hyperplanes for each combination of delta and $\varepsilon$.
%% delta spans the interval from 0.05 to 0.3, while $\varepsilon$ from 0.0006 to 0.0050.  % it's in the table
%For the aircraft problem, the domain of X is bounded in the range $X_{0} \in
%(-10000, 10000)$, $X_{1}\in (-734, 734)$, $X_{2}\in (-10000, 10000)$ and
%$X_{3}\in (-790, 790)$. MATLAB returns a controller with 27 regions and 217
%hyperplanes, except when delta=0.30 and $\varepsilon$=0.001 where the controller
%has 26 regions and 208 hyperplanes. In general, when the disturbance value
%increases, the state domain X shrinks to be robust against a greater
%disturbance, or the number of regions reduces.

We do not compare against the existing technique of~\citet{imperialrmpc} which aims to reduce memory
usage in explicit MPC control, as it requires the user to provide a (uniform)
fixed-precision up front. Instead, we let Daisy find the minimum uniform
precision needed. Additionally, we compare against a uniform 32-bit precision
baseline, which in the absence of special user insight would be a reasonable safe choice.

\autoref{tab:ipd} shows the total number of bits required to implement each
controller using different precision options: uniform 32 bit precision
(`Uni32'), minimal uniform precision (`Uni', chosen precision in parentheses)
and mixed precision (`Mix'). We split the memory requirement into the bits
required for storing $F$, $G$, $H$ and $K$. We show the results for each benchmark
for ten different combinations of $\Delta$ and $\varepsilon_Q$, varying one while
keeping the other fixed.
For the pendulum, the execution time of the whole analysis is 10 minutes no matter the
initial values for $\Delta$ and $\varepsilon_Q$, while for the aircraft it is 36
minutes. 


%\rocco{maybe table for execution times}
% In \autoref{tab:ipd} we report the evaluation of our approach on both the
% inverted pendulum problem described in the motivation section, and a well-known
% 4D example for aircraft controller design. (Describe aircraft). For each
% benchmark, we report the results for ten different combinations of delta and
% $\varepsilon$. We divided them in two halves: in the  first, we fix the value
% for delta and we try different $\varepsilon$ combinations, while in the second
% halve we do the opposite.

% \eva{In the following, I'd discuss the two examples together and structure the 
% discussion as follows:
% \begin{itemize}
% 	\item give the constant info (number of regions etc.) for both benchmarks
% 	\item explain the variation of delta and eps (ideally with a motivation for why this is interesting)
% 	\item discuss the improvements we get (i.e. first discuss the metric we care about most)
% 	\item discuss other interesting details
% \end{itemize}
% Discussing the two examples together let's us compare their results easier, i.e.
% point out differences or commonalities.}

%\eva{Fix Table 1 caption. It's not the difference, but savings. Difference is absolute.}
%\eva{Compute the actual averages for all F, G, H, K and not only approximations}
For the inverted pendulum, minimal uniform precision saves on average $52.6\%$ of memory
compared to a uniform 32 bit baseline overall, i.e. for $F$, $G$, $H$ and $K$ together. 
Mixed-precision further reduces the memory
requirement by $10.5\%$ on average with respect to the minimal uniform baseline (and 57.5\% w.r.t to 
uniform 32 bit baseline). 
\autoref{tab:ipd} shows a more detailed breakdown of the memory requirements and savings
between $F$, $G$, $H$ and $K$.
The memory requirements for the storage of hyperplanes $H$ and $K$ depends only on
the size of the tubes $\varepsilon_Q$ so that memory requirements remain constant
for fixed $\varepsilon_Q$.

%\eva{Compute actual and overall averages here too}
For the aircraft example, minimal uniform precision saves on average $19.3\%$
overall w.r.t. a uniform 32 bit baseline, and mixed-precision saves an
additional $17.7\%$ w.r.t. minimal uniform precision (33.6\% w.r.t to uniform 32 bit). We observe higher relative
memory savings by mixed-precision for storing $H$ and $K$ than for the inverted pendulum example.

% we save 20\% (in average) of memory when activation functions
% are in minimal uniform precision, and an additional 7\% (in average) when we use
% mixed precision. For the storage of hyperplanes equations, we save more than
% 19\% of memory (in average) from uniform precision in 32bits to minimal uniform
% precision, and an additional 19\% (in average) from uniform to mixed precision.

% We use Matlab's MPT toolbox~\cite{matlabMPT} to compute the
% explicit MPC, while underlying computations rely on YALMIP(\cite{matlabYALMIP}).
%We evaluate our algorithm with different choices of $\varepsilon_Q$ and $\Delta$.

For the inverted pendulum MPT toolbox~\cite{matlabMPT} computes a controller
consisting of $14$ regions with $58$ 2D hyperplanes in total for all choices of $\varepsilon_Q$ and $\Delta$. 
For the aircraft model, Matlab computes a robust EMPC with
$27$ regions and $217$ 4D hyperplanes for most values of $\varepsilon_Q$ and $\Delta$. 
The only exception is when $\Delta=0.30$
and $\varepsilon_Q=0.001$ for which we get $26$ regions having $208$ hyperplanes.
In general, we expect that increasing $\Delta$ results in shrinking of feasible
set size.

%We noticed the following common behavior in the two benchmarks: 
As expected, when $\Delta$ decreases, the control actions ($F$, $G$) need to be implemented
more precisely and require more memory, because the space for approximation error
is reduced. Similarly, when the value of $\varepsilon_Q$ increases, the memory
requirements for $H$ and $K$ can be relaxed.

%We note two differences among these two benchmarks.
We note that for the aircraft example,
the precision for $F$ and $G$ is almost double with respect to the pendulum. This is
because the magnitude of $F$ and $G$ is on the order of $10^{3}$ while for the
pendulum it is on the order of several units. Note, however, that for $F$ and $G$ the memory gain
from minimal uniform to mixed precision (\%UvM) is only slightly less than the one for the pendulum.


%The second difference consists in the last two lines in Table\ref{tab:ipd}.
For the aircraft example, when $\varepsilon_Q=0.0030$, the error due to selecting the wrong region (\maxUij) 
exceeds the given value of $\Delta=0.1$ and our algorithm
needs to design a new controller (corresponding to the \texttt{else} branch in~\autoref{lst:alg}). 
The loop converges after 5 iterations (\texttt{then}
branch in our algorithm) with $\Delta=0.112$. In each iteration, the value
for $\Delta$ is increased by $\varepsilon_{SAFE} = 10^{-3}$.
Thus, the algorithm reduces memory demand
at the expense of slightly more disturbance for the controller. When
$\varepsilon_Q=0.0050$ the analysis converges after 4 iterations.
For the other values of $\Delta$ and $\varepsilon$ and for the pendulum example,
the loop in~\autoref{lst:alg} is executed only once.

% \eva{It is still not clear to me whether we need this discussion or the columns `max' and `$err(u_i)$'
% in the table. These are internal computations and it seems that they do not really show anything
% surprising/different than the number of bits show?}
% The first halves of both pendulum and aircraft sections in \autoref{tab:ipd},
% show how $\Delta$ does not affect the maximal error among regions \maxUij. We
% verify that a variation for $\Delta$ (typically in the order of $\pm$ 0.1)
% results in a minimal alteration to the domain of regions X (in the order of $\pm
% 10^{-10}$), not enough to produce a sensitive variation to \maxUij. Then, when
% \maxUij\space is constant the error bound for computations $err(u_{i})$
% decreases together with $\Delta$. The memory required for the storage of F and G
% depends on $err(u_{i})$: the smaller the value for $err(u_{i})$, the more
% precise has to be the arithmetic for activation functions.

% In the second halves in \autoref{tab:ipd}, we fix the value for $\Delta$ and we
% try different $\varepsilon_Q$. When we increase the safe space between two regions
% by $\varepsilon_Q$, the value for \maxUij is computed for new corner points. It
% happens because of the continuity of PWA activation functions: when
% \statevarmath is on the border between two generic regions $i$, $j$ the value
% for \maxUij is close to zero. The more $\varepsilon_Q$ moves \statevarmath far
% from the border, the more \maxUij increases. When \maxUij is almost equal to
% $\Delta$, the space for $err(u_{i})$ is minimized, and the controller requires
% high precision for computations (see last lines for pendulum and aircraft in
% Table\ref{tab:ipd}). On the other hand, when $\varepsilon_Q$ increases, the
% analysis can reduce the memory requirements for the storage of borders (see
% columns Uni and Mix in the second halves of both benchmarks).




\subsection{Scalability}
%\Mahmoud{
The goal of the experiment in this section is to show that our algorithm works
well even for the case that the controller consists of thousands of regions.
%Using the algorithm described in Section \ref{???}, the minimum number of bits for storing the controller is computed such that $\delta$ and $\varepsilon$ are respected. 
This experiment is different from the end-to-end case in the sense that
designing robust EMPC for the error bound $\Delta$ is replaced with designing
EMPC that might not account for $\Delta$. Given an EMPC, one can perform
robustness analysis to come up with an input error bound $\Delta$ under which
the performance specifications are satisfied. From the implementation point of
view, this helps us to generate controllers with thousands of regions,
skiping the restrictions for designing robust EMPC with longer time horizons. 
% At this point, our algorithm takes $\Delta$ (along with a properly selected
% $\varepsilon_Q$) and produces the optimum mixed precision that results in
% significant memory save.

The benchmark in this experiment is the double integrator, a canonical example
of a second order control system. The state space description
for the discrete time version of the double integrator is given by:
\begin{equation}
x_{k+1}=
\begin{bmatrix}
1 & T_s\\
0& 1	
\end{bmatrix}
x_k+
\begin{bmatrix}
0\\
T_s
\end{bmatrix}u_k
\label{eq:int2_ss}
\end{equation}
where, $T_s=0.1$ is the sample time. The state and input constraints are
\begin{align*}
	\begin{bmatrix}
		-5\\-5
	\end{bmatrix}\leq&
	x_k\leq
	\begin{bmatrix}
	5\\5
	\end{bmatrix}, \quad
	-1\leq u_k \leq 1
\end{align*}%}
%\eva{More details about this + citation}


For this example, $\Delta$ and $\varepsilon_Q$ are fixed to $0.1$ and $0.001$,
respectively. By changing the time horizon $N$, we evaluate the scalability
of our approach for controllers with very large number of regions, as 
increasing $N$ generally results in a higher number of regions for the output controller. To
compute the explicit model predictive controllers for different time horizons,
we use Matlab's MPC toolbox. %\todo{is this the same as before? If so, add a 'again'}
 
%For this example, we fix the values for $\delta$ to $0.1$ and $\varepsilon$ to $0.001$,
%but we consider also the prediction horizon $N$, as an input parameter for the
%analysis. At design time, the prediction horizon of the controller m models how
%many future iterations of the feed-forward system are considered in the
%optimization problem (infinite horizon would correspond to the optimal
%controller). Note that in this example, the number of regions is highly
%correlated with the prediction horizon m of the controller: the number of regions
%increases with larger m, and thus also increases the execution time of our analysis.
%Even if this is a general behavior, it is not the rule.
%\eva{What does the above sentence mean?}

The design of the controller in Matlab takes a few minutes, then controllers and
hyperplanes are encoded in Daisy for finite-precision assignment. On average, the
analysis of a single controller or hyperplane in Daisy requires less than two
minutes and less than 500MB of memory. The finite-precision assignment is trivially parallelizable, 
because any controller or hyperplane can be analyzed independently.
We thus run this analysis on a cluster with 48 cores, and note that the 
memory consumption remains below 15GB. 
%\eva{There should be an explanation why this is not providing end-to-end robustness}


\autoref{tab:di} shows the results of this experiment.
We observe that our analysis scales up to 1829 controllers and 14632 hyperplanes in a few hours.  The computed minimal uniform precision saves on average $44.5\%$ of memory
compared to a uniform $32$ bit baseline overall (for $F$, $G$, $H$ and $K$ together). 
Moreover, mixed-precision reduces the memory requirement by an additional
$9.3\%$ on average with respect to the minimal uniform baseline (49.6\% w.r.t. uniform 32 bit).
We further observe that relative savings remain largely constant for different 
prediction horizons.
% \autoref{tab:di} shows the stability of memory tuning in Daisy is
% constant with respect to the number of controllers or hyperplanes. This is true
% for both F, G and H, K. Indeed, the memory gain from uniform in 32bits and
% minimal uniform (\%32vU) is constant among different input values m. The same
% holds for (\%UvM).